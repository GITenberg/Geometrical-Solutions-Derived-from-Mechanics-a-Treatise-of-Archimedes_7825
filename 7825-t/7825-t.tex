% %%%%%%%%%%%%%%%%%%%%%%%%%%%%%%%%%%%%%%%%%%%%%%%%%%%%%%%%%%%%%%%%%%%%%%% %
%                                                                         %
% The Project Gutenberg EBook of Geometrical Solutions Derived from Mechanics, by 
% Archimedes                                                              %
%                                                                         %
% This eBook is for the use of anyone anywhere in the United States and most
% other parts of the world at no cost and with almost no restrictions     %
% whatsoever.  You may copy it, give it away or re-use it under the terms of
% the Project Gutenberg License included with this eBook or online at     %
% www.gutenberg.org.  If you are not located in the United States, you'll have
% to check the laws of the country where you are located before using this ebook.
%                                                                         %
%                                                                         %
%                                                                         %
% Title: Geometrical Solutions Derived from Mechanics                     %
%        A Treatise of Archimedes                                         %
%                                                                         %
% Author: Archimedes                                                      %
%                                                                         %
% Release Date: November 1, 2014 [EBook #7825]                            %
%                                                                         %
% Language: English                                                       %
%                                                                         %
% Character set encoding: ISO-8859-1                                      %
%                                                                         %
% *** START OF THIS PROJECT GUTENBERG EBOOK GEOMETRICAL SOLUTIONS ***     %
%                                                                         %
% %%%%%%%%%%%%%%%%%%%%%%%%%%%%%%%%%%%%%%%%%%%%%%%%%%%%%%%%%%%%%%%%%%%%%%% %

\def\ebook{7825}
%%%%%%%%%%%%%%%%%%%%%%%%%%%%%%%%%%%%%%%%%%%%%%%%%%%%%%%%%%%%%%%%%%%%%%
%%                                                                  %%
%% Packages and substitutions:                                      %%
%%                                                                  %%
%% book:     Required.                                              %%
%% inputenc: Latin-1 text encoding. Required.                       %%
%%                                                                  %%
%% ifthen:   Logical conditionals. Required.                        %%
%%                                                                  %%
%% amsmath:  AMS mathematics enhancements. Required.                %%
%% amssymb:  Additional mathematical symbols. Required.             %%
%%                                                                  %%
%% alltt:    Fixed-width font environment. Required.                %%
%%                                                                  %%
%% indentfirst: Indent first line of each section. Optional.        %%
%%                                                                  %%
%% graphicx: Standard interface for graphics inclusion. Required.   %%
%%                                                                  %%
%% calc:     Length calculations. Required.                         %%
%%                                                                  %%
%% fancyhdr: Enhanced running headers and footers. Required.        %%
%%                                                                  %%
%% geometry: Enhanced page layout package. Required.                %%
%% hyperref: Hypertext embellishments for pdf output. Required.     %%
%%                                                                  %%
%%                                                                  %%
%% Producer's Comments:                                             %%
%%                                                                  %%
%% Originally produced in 2003 by Gordon Keener <gkeener@nc.rr.com> %%
%%                                                                  %%
%% Compilation Flags:                                               %%
%%                                                                  %%
%%   The following behavior may be controlled by boolean flags.     %%
%%                                                                  %%
%%   ForPrinting (false by default):                                %%
%%   If false, compile a screen optimized file (one-sided layout,   %%
%%   blue hyperlinks). If true, print-optimized PDF file: Larger    %%
%%   text block, two-sided layout, black hyperlinks.                %%
%%                                                                  %%
%%                                                                  %%
%% PDF pages:  59 (if ForPrinting set to false)                     %%
%% PDF page size: 4.5 x 6.5" (non-standard)                         %%
%%                                                                  %%
%% Summary of log file:                                             %%
%% * No warnings.                                                   %%
%%                                                                  %%
%% Compile History:                                                 %%
%%                                                                  %%
%% November, 2014: (Andrew D. Hwang)                                %%
%%             texlive2009, GNU/Linux                               %%
%%                                                                  %%
%% Command block:                                                   %%
%%                                                                  %%
%%     pdflatex x2                                                  %%
%%                                                                  %%
%%                                                                  %%
%% November 2014: pglatex.                                          %%
%%   Compile this project with:                                     %%
%%   pdflatex 7825-t.tex ..... TWO times                            %%
%%                                                                  %%
%%   pdfTeX, Version 3.1415926-2.5-1.40.14 (TeX Live 2013/Debian)   %%
%%                                                                  %%
%%%%%%%%%%%%%%%%%%%%%%%%%%%%%%%%%%%%%%%%%%%%%%%%%%%%%%%%%%%%%%%%%%%%%%
\listfiles
\documentclass[12pt,leqno]{book}[2005/09/16]

%%%%%%%%%%%%%%%%%%%%%%%%%%%%% PACKAGES %%%%%%%%%%%%%%%%%%%%%%%%%%%%%%%
\usepackage[latin1]{inputenc}[2006/05/05]

\usepackage{ifthen}[2001/05/26]  %% Logical conditionals

\usepackage[greek,english]{babel}

\usepackage{amsmath}[2000/07/18] %% Displayed equations
\usepackage{amssymb}[2002/01/22] %% and additional symbols

\usepackage{alltt}[1997/06/16]   %% boilerplate, credits, license

\IfFileExists{indentfirst.sty}{%
  \usepackage{indentfirst}[1995/11/23]
}{}

\usepackage{graphicx}[1999/02/16]%% For diagrams

\usepackage{calc}[2005/08/06]

\usepackage{fancyhdr} %% For running heads

%%%%%%%%%%%%%%%%%%%%%%%%%%%%%%%%%%%%%%%%%%%%%%%%%%%%%%%%%%%%%%%%%
%%%% Interlude:  Set up PRINTING (default) or SCREEN VIEWING %%%%
%%%%%%%%%%%%%%%%%%%%%%%%%%%%%%%%%%%%%%%%%%%%%%%%%%%%%%%%%%%%%%%%%

% ForPrinting=true                     false (default)
% Asymmetric margins                   Symmetric margins
% 1 : 1.6 text block aspect ratio      3 : 4 text block aspect ratio
% Black hyperlinks                     Blue hyperlinks
% Start major marker pages recto       No blank verso pages
%
\newboolean{ForPrinting}

%% UNCOMMENT the next line for a PRINT-OPTIMIZED VERSION of the text %%
%\setboolean{ForPrinting}{true}

%% Initialize values to ForPrinting=false
\newcommand{\Margins}{hmarginratio=1:1}     % Symmetric margins
\newcommand{\HLinkColor}{blue}              % Hyperlink color
\newcommand{\PDFPageLayout}{SinglePage}
\newcommand{\TransNote}{Transcriber's Note}
\newcommand{\TransNoteCommon}{%
  The camera-quality files for this public-domain ebook may be
  downloaded \textit{gratis} at
  \begin{center}
    \texttt{www.gutenberg.org/ebooks/\ebook}.
  \end{center}

  This ebook was produced using scanned images generously provided by
  the University of California, Berkeley, Mathematics and Statistics
  Library, through the Internet Archive.
  \bigskip

  Minor typographical corrections and presentational changes have been
  made without comment.
  \bigskip
}

\newcommand{\TransNoteText}{%
  \TransNoteCommon

  This PDF file is optimized for screen viewing, but may be recompiled
  for printing. Please consult the preamble of the \LaTeX\ source file
  for instructions and other particulars.
}
%% Re-set if ForPrinting=true
\ifthenelse{\boolean{ForPrinting}}{%
  \renewcommand{\Margins}{hmarginratio=2:3} % Asymmetric margins
  \renewcommand{\HLinkColor}{black}         % Hyperlink color
  \renewcommand{\PDFPageLayout}{TwoPageRight}
  \renewcommand{\TransNote}{Transcriber's Note}
  \renewcommand{\TransNoteText}{%
    \TransNoteCommon

    This PDF file is optimized for printing, but may be recompiled for
    screen viewing. Please consult the preamble of the \LaTeX\ source
    file for instructions and other particulars.
  }
}{% If ForPrinting=false, don't skip to recto
  \renewcommand{\cleardoublepage}{\clearpage}
}
%%%%%%%%%%%%%%%%%%%%%%%%%%%%%%%%%%%%%%%%%%%%%%%%%%%%%%%%%%%%%%%%%
%%%%  End of PRINTING/SCREEN VIEWING code; back to packages  %%%%
%%%%%%%%%%%%%%%%%%%%%%%%%%%%%%%%%%%%%%%%%%%%%%%%%%%%%%%%%%%%%%%%%

\ifthenelse{\boolean{ForPrinting}}{%
  \setlength{\paperwidth}{8.5in}%
  \setlength{\paperheight}{11in}%
% 1:1.6
  \usepackage[body={5in,8in},\Margins]{geometry}[2002/07/08]
}{%
  \setlength{\paperwidth}{4.5in}%
  \setlength{\paperheight}{6.5in}%
  \raggedbottom
% 3:4
  \usepackage[body={4.25in,5.6in},\Margins,includeheadfoot]{geometry}[2002/07/08]
}

\providecommand{\ebook}{00000}    % Overridden during white-washing
\usepackage[pdftex,
  hyperfootnotes=false,
  pdftitle={The Project Gutenberg eBook \#\ebook: Geometrical Solutions Derived from Mechanics.},
  pdfauthor={Archimedes of Syracuse}
  pdfkeywords={The Internet Archive, Gordon Keener},
  pdfstartview=Fit,    % default value
  pdfstartpage=1,      % default value
  pdfpagemode=UseNone, % default value
  bookmarks=true,      % default value
  linktocpage=false,   % default value
  pdfpagelayout=\PDFPageLayout,
  pdfdisplaydoctitle,
  pdfpagelabels=true,
  bookmarksopen=true,
  bookmarksopenlevel=0,
  colorlinks=true,
  linkcolor=\HLinkColor]{hyperref}[2007/02/07]


%% Fixed-width environment to format PG boilerplate %%
\newenvironment{PGtext}{%
\begin{alltt}
\fontsize{8.1}{10}\ttfamily\selectfont}%
{\end{alltt}}

%% Miscellaneous global parameters %%
% No hrule in page header
\renewcommand{\headrulewidth}{0pt}

% Loosen spacing
\setlength{\emergencystretch}{1em}

% [** TN: Changes made for stylistic consistency]
\newcommand{\Chg}[2]{#2}

%% Running heads %%
\newcommand{\FlushRunningHeads}{\clearpage\fancyhf{}}
\newcommand{\InitRunningHeads}{%
  \setlength{\headheight}{15pt}
  \pagestyle{fancy}
  \thispagestyle{empty}
  \ifthenelse{\boolean{ForPrinting}}
             {\fancyhead[RO,LE]{\thepage}}
             {\fancyhead[R]{\thepage}}
}

% Uniform style for running heads
\ifthenelse{\boolean{ForPrinting}}{%
  \newcommand{\HeadSize}{\normalsize}%
}{
  \newcommand{\HeadSize}{\footnotesize}%
}
\newcommand{\RHeads}[1]{\HeadSize\textsc{\MakeLowercase{#1}}}

\newcommand{\SetRunningHeads}[2][Geometrical Solutions Derived from Mechanics.]{%
  \fancyhead[CE]{\RHeads{#1}}
  \fancyhead[CO]{\RHeads{#2}}
}

\newcommand{\BookMark}[2]{\phantomsection\pdfbookmark[#1]{#2}{#2}}

%% Major document divisions %%
\newcommand{\PGBoilerPlate}{%
  \pagenumbering{Alph}
  \pagestyle{empty}
  \BookMark{0}{PG Boilerplate.}
}
\newcommand{\FrontMatter}{%
  \cleardoublepage
  \frontmatter
  \BookMark{-1}{Front Matter.}
}
\newcommand{\MainMatter}{%
  \FlushRunningHeads
  \InitRunningHeads
  \mainmatter
  \BookMark{-1}{Main Matter.}
}
\newcommand{\BackMatter}{%
  \FlushRunningHeads
  \InitRunningHeads
  \backmatter
  \BookMark{-1}{Back Matter.}
}
\newcommand{\PGLicense}{%
  \FlushRunningHeads
  \pagenumbering{Roman}
  \InitRunningHeads
  \BookMark{-1}{PG License.}
  \SetRunningHeads[License.]{License.}
}

%% Sectional units %%
% Typographical abstraction
\newcommand{\SectTitle}[1]{%
  \section*{\centering\normalsize\normalfont #1}
}

\newcommand{\Section}[1]{
  \FlushRunningHeads
  \InitRunningHeads
  \SetRunningHeads{#1}
  \BookMark{0}{#1}
  \setcounter{footnote}{0}
  \SectTitle{\MakeUppercase{#1}}
}

\newcommand{\Subsection}[1]{
  \BookMark{1}{Section #1}
  \SectTitle{#1}
}

%% Diagrams %%
\newcommand{\Graphic}[2]{%
  \includegraphics[width=#1]{./images/#2.png}%
}
% \Figure[width]{figure number}{file}
\newcommand{\DefWidth}{0.66\textwidth}% Default figure width
\newcommand{\Figure}[3][\DefWidth]{%
  \begin{figure}[hbt!]
    \centering
    \phantomsection\label{fig:#2}
    \Graphic{#1}{#3} \\
    Fig.~#2.
  \end{figure}\ignorespaces%
}
% ** One-use macro for side-by-side diagrams
\newcommand{\Figures}[4]{%
  \begin{figure}[hbt!]
    \centering
    \phantomsection\label{fig:#1}
    \Graphic{0.45\textwidth}{#2}\hfill
    \phantomsection\label{fig:#3}
    \Graphic{0.45\textwidth}{#4}
  \end{figure}\ignorespaces%
}

% Figure labels
\newcommand{\Fig}[1]{\hyperref[fig:#1]{Fig.~{\upshape #1}}}

% Page separators
\newcommand{\PageSep}[1]{\ignorespaces}

% Miscellaneous textual conveniences (N.B. \emph, not \textit)
\newcommand{\ie}{i.\;e.}
\newcommand{\ibid}{\emph{ibid.}}
\newcommand{\Ibid}{\emph{Ibid.}}
\newcommand{\QED}{Q.\,E.\,D.}

\newcommand{\SameMark}{\addtocounter{footnote}{-1}\footnotemark}

\newcommand{\First}[1]{\textsc{#1}}
\newcommand{\Signature}[2]{%
  \null\hfill\textsc{#1}\qquad\null \\
  \null\qquad\textsc{#2}
}

\newcommand{\Title}[1]{\textit{#1}}

%% Miscellaneous mathematical formatting %%
\DeclareMathOperator{\cone}{cone\,}
\DeclareMathOperator{\cones}{cones\,}
\DeclareMathOperator{\cylinder}{cylinder\,}
\DeclareMathOperator{\sphere}{sphere\,}
\DeclareMathOperator{\spheres}{spheres\,}
\DeclareMathOperator{\spheroid}{spheroid\,}
\DeclareMathOperator{\segment}{segment\,}
\DeclareMathOperator{\segm}{segm.\,}
\DeclareMathOperator{\Tri}{triangle\,}
\DeclareInputMath{183}{\cdot}

\renewcommand{\|}{\parallel}

%%%%%%%%%%%%%%%%%%%%%%%% START OF DOCUMENT %%%%%%%%%%%%%%%%%%%%%%%%%%
\begin{document}
%% PG BOILERPLATE %%
\PGBoilerPlate
\begin{center}
\begin{minipage}{\textwidth}
\small
\begin{PGtext}
The Project Gutenberg EBook of Geometrical Solutions Derived from Mechanics, by 
Archimedes

This eBook is for the use of anyone anywhere in the United States and most
other parts of the world at no cost and with almost no restrictions
whatsoever.  You may copy it, give it away or re-use it under the terms of
the Project Gutenberg License included with this eBook or online at
www.gutenberg.org.  If you are not located in the United States, you'll have
to check the laws of the country where you are located before using this ebook.



Title: Geometrical Solutions Derived from Mechanics
       A Treatise of Archimedes

Author: Archimedes

Release Date: November 1, 2014 [EBook #7825]

Language: English

Character set encoding: ISO-8859-1

*** START OF THIS PROJECT GUTENBERG EBOOK GEOMETRICAL SOLUTIONS ***
\end{PGtext}
\end{minipage}
\end{center}
\newpage
%% Credits and transcriber's note %%
\begin{center}
\begin{minipage}{\textwidth}
\begin{PGtext}
Produced by Gordon Keener
\end{PGtext}
\end{minipage}
\vfill
\end{center}

\begin{minipage}{0.85\textwidth}
\small
\BookMark{0}{Transcriber's Note.}
\subsection*{\centering\normalfont\scshape%
\normalsize\MakeLowercase{\TransNote}}%

\raggedright
\TransNoteText
\end{minipage}
%%%%%%%%%%%%%%%%%%%%%%%%%%% FRONT MATTER %%%%%%%%%%%%%%%%%%%%%%%%%%
\PageSep{i}
\FrontMatter
\begin{center}
\LARGE GEOMETRICAL SOLUTIONS \\[12pt]
\normalsize DERIVED FROM \\
\LARGE MECHANICS
\vfill

\large A TREATISE OF ARCHIMEDES
\vfill

\footnotesize RECENTLY DISCOVERED AND TRANSLATED~FROM~THE~GREEK~BY \\
\normalsize DR.~J.~L. HEIBERG \\
\scriptsize PROFESSOR OF CLASSICAL PHILOLOGY AT~THE~UNIVERSITY~OF~COPENHAGEN
\normalsize
\vfill

\footnotesize WITH AN INTRODUCTION BY \\
\small DAVID EUGENE SMITH \\
\scriptsize PRESIDENT OF TEACHERS COLLEGE, COLUMBIA~UNIVERSITY,~NEW~YORK
\vfill

\footnotesize ENGLISH VERSION TRANSLATED FROM~THE~GERMAN BY~LYDIA~G.~ROBINSON \\
AND REPRINTED FROM ``THE MONIST,'' APRIL,~1909
\vfill\vfill\vfill

CHICAGO \\
THE OPEN COURT PUBLISHING COMPANY \\
LONDON AGENTS \\
KEGAN PAUL, TRENCH, TR�BNER \& CO., LTD. \\
1909
\end{center}
\PageSep{ii}
\newpage
\begin{center}
\null\vfill
\footnotesize\scshape
Copyright by \\
The Open Court Publishing Co. \\
1909
\vfill
\end{center}
\PageSep{1}
\MainMatter

\Section{Introduction.}

\First{If} there ever was a case of appropriateness in discovery,
the finding of this manuscript in the summer of~1906
was one. In the first place it was appropriate that the discovery
should be made in Constantinople, since it was here
that the West received its first manuscripts of the other extant
works, nine in number, of the great Syracusan.
It was furthermore appropriate that the discovery should be made
by Professor Heiberg, \textit{facilis princeps} among all workers
in the field of editing the classics of Greek mathematics,
and an indefatigable searcher of the libraries of Europe
for manuscripts to aid him in perfecting his labors.  And
finally it was most appropriate that this work should appear
at a time when the affiliation of pure and applied
mathematics is becoming so generally recognized all over
the world. We are sometimes led to feel, in considering
isolated cases, that the great contributors of the past have
worked in the field of pure mathematics alone, and the
saying of Plutarch that Archimedes felt that ``every kind
of art connected with daily needs was ignoble and vulgar''\footnote
  {Marcellus, 17.}
may have strengthened this feeling. It therefore assists
us in properly orientating ourselves to read another treatise
from the greatest mathematician of antiquity that sets
clearly before us his indebtedness to the mechanical applications
of his subject.

Not the least interesting of the passages in the manuscript
\PageSep{2}
is the first line, the greeting to Eratosthenes. It is
well known, on the testimony of Diodoros his countryman,
that Archimedes studied in Alexandria, and the latter frequently
makes mention of Konon of Samos whom he knew
there, probably as a teacher, and to whom he was indebted
for the suggestion of the spiral that bears his name. It is
also related, this time by Proclos, that Eratosthenes was a
contemporary of Archimedes, and if the testimony of so
late a writer as Tzetzes, who lived in the twelfth century,
may be taken as valid, the former was eleven years the
junior of the great Sicilian. Until now, however, we have
had nothing definite to show that the two were ever acquainted.
The great Alexandrian savant,---poet, geographer,
arithmetician,---affectionately called by the students
Pentathlos, the champion in five sports,\footnote
  {His nickname of \textit{Beta} is well known, possibly because his lecture room
  was number~2.}
selected by
Ptolemy Euergetes to succeed his master, Kallimachos the
poet, as head
of the great Library,---this man, the most
renowned of his time in Alexandria, could hardly have
been a teacher of Archimedes, nor yet the fellow student of
one who was so much his senior. It is more probable that
they were friends in the later days when Archimedes was
received as a savant rather than as a learner, and this is
borne out by the statement at the close of proposition~I
which refers to one of his earlier works, showing that this
particular treatise was a late one. This reference being
to one of the two works dedicated to Dositheos of Kolonos,\footnote
  {We know little of his works, none of which are extant. Geminos and
   Ptolemy refer to certain observations made by him in 200~B.~C., twelve years
   after the death of Archimedes. Pliny also mentions him.}
and one of these (\Title{De lineis spiralibus}) referring to an
earlier treatise sent to Konon,\footnote
  {\selectlanguage{greek} T\~wn pot\`i K\'onwna \'apustal\'entwn jewrhm\'atwn.}
we are led to believe that
this was one of the latest works of Archimedes and that
Eratosthenes was a friend of his mature years, although
\PageSep{3}
one of long standing. The statement that the preliminary
propositions were sent ``some time ago'' bears out this idea
of a considerable duration of friendship, and the idea that
more or less correspondence had resulted from this communication
may be inferred by the statement that he saw,
as he had previously said, that Eratosthenes was ``a capable
scholar and a prominent teacher of philosophy,'' and also
that he understood ``how to value a mathematical method
of investigation when the opportunity offered.'' We have,
then, new light upon the relations between these two men,
the leaders among the learned of their day.

A second feature of much interest in the treatise is the
intimate view that we have into the workings of the mind
of the author. It must always be remembered that Archimedes
was primarily a discoverer, and not primarily a compiler
as were Euclid, Apollonios, and Nicomachos. Therefore
to have him follow up his first communication of theorems
to Eratosthenes by a statement of his mental processes
in reaching his conclusions is not merely a contribution
to mathematics but one to education as well. Particularly
is this true in the following statement, which may well be
kept in mind in the present day: ``I have thought it well
to analyse and lay down for you in this same book a peculiar
method by means of which it will be possible for you
to derive instruction as to how certain mathematical questions
may be investigated by means of mechanics. And I
am convinced that this is equally profitable in demonstrating
a proposition itself; for much that was made evident
to me through the medium of mechanics was later proved
by means of geometry, because the treatment by the former
method had not yet been established by way of a demonstration.
For of course it is easier to establish a proof if one
has in this way previously obtained a conception of the
questions, than for him to seek it without such a preliminary
notion\dots. Indeed I assume that some one among the
\PageSep{4}
investigators of to-day or in the future will discover by the
method here set forth still other propositions which have
not yet occurred to us.'' Perhaps in all the history of
mathematics no such prophetic truth was ever put into
words. It would almost seem as if Archimedes must have
seen as in a vision the methods of Galileo, Cavalieri, Pascal,
Newton, and many of the other great makers of the mathematics
of the Renaissance and the present time.

The first proposition concerns the quadrature of the parabola,
a subject treated at length in one of his earlier
communications to Dositheos.\footnote
  {\selectlanguage{greek} Tetragwnismds parabol\~hs.}
He gives a digest of the
treatment, but with the warning that the proof is not complete,
as it is in his special work upon the subject. He has,
in fact, summarized propositions VII--XVII of his communication
to Dositheos, omitting the geometric treatment
of propositions XVIII--XXIV.  One thing that he
does not state, here or in any of his works, is where the
idea of center of gravity\footnote
  {\selectlanguage{greek} K\'entra bar\~wn, \selectlanguage{english} for ``barycentric'' is a very old term.}
started. It was certainly a common
notion in his day, for he often uses it without defining
it. It appears in Euclid's\footnote
  {At any rate in the anonymous fragment \Title{De levi et ponderoso}, sometimes
  attributed to him.}
time, but how much earlier we
cannot as yet say.

Proposition~II states no new fact. Essentially it means
that if a sphere, cylinder, and cone (always circular) have
the same radius,~$r$, and the altitude of the cone is~$r$ and that
of the cylinder~$2r$, then the volumes will be as $4 : 1 : 6$,
which is true, since they are respectively $\frac{4}{3}\pi r^{3}$, $\frac{1}{3}\pi r^{3}$, and
$2\pi r^{3}$. The interesting thing, however, is the method pursued,
the derivation of geometric truths from principles
of mechanics. There is, too, in every sentence, a little
suggestion of Cavalieri, an anticipation by nearly two thousand
years of the work of the greatest immediate precursor
of Newton. And the geometric imagination that Archimedes
\PageSep{5}
shows in the last sentence is also noteworthy as one
of the interesting features of this work: ``After I had thus
perceived that a sphere is four times as large as the cone\dots
it occurred to me that the surface of a sphere is four times
as great as its largest circle, in which I proceeded from the
idea that just as a circle is equal to a triangle whose base is
the periphery of the circle, and whose altitude is equal to
its radius, so a sphere is equal to a cone whose base is the
same as the surface of the sphere and whose altitude is
equal to the radius of the sphere.'' As a bit of generalization
this throws a good deal of light on the workings of
Archimedes's mind.

In proposition~III he considers the volume of a spheroid,
which he had already treated more fully in one of his
letters to Dositheos,\footnote
  {\selectlanguage{greek} Per\`i kwnoeide\~wn kai sfairoeide\~wn.}
and which contains nothing new from
a mathematical standpoint. Indeed it is the method rather
than the conclusion that is interesting in such of the subsequent
propositions as relate to mensuration. Proposition~V
deals with the center of gravity of a segment of a conoid, and
proposition~VI with the center of gravity of a hemisphere,
thus carrying into solid geometry the work of Archimedes
on the equilibrium of planes and on their centers of gravity.\footnote
  {\selectlanguage{greek} 'Epip\'edwn \`isorropi\~wn \^h k\'entra bar\~wn \'epip\'edwn.}
The general method is that already known in the
treatise mentioned, and this is followed through proposition~X\@.

Proposition~XI is the interesting case of a segment of
a right cylinder cut off by a plane through the center of
the lower base and tangent to the upper one. He shows
this to equal one-sixth of the square prism that circumscribes
the cylinder. This is well known to us through the
formula $v = 2r^{2}h/3$, the volume of the prism being~$4r^{2}h$,
and requires a knowledge of the center of gravity of the
cylindric section in question.  Archimedes is, so far as we
\PageSep{6}
know, the first to state this result, and he obtains it by his
usual method of the skilful balancing of sections. There
are several lacunae in the demonstration, but enough of
it remains to show the ingenuity of the general plan. The
culminating interest from the mathematical standpoint lies
in proposition~XIII, where Archimedes reduces the whole
question to that of the quadrature of the parabola. He
shows that a fourth of the circumscribed prism is to the
segment of the cylinder as the semi-base of the prism is to
the parabola inscribed in the semi-base; that is, that
$\frac{1}{4}p : v = \frac{1}{2}b : (\frac{2}{3} � \frac{1}{2}b)$,
whence $v = \frac{1}{6}p$. Proposition~XIV is incomplete,
but it is the conclusion of the two preceding propositions.

In general, therefore, the greatest value of the work
lies in the following:

1. It throws light upon the hitherto only suspected relations
of Archimedes and Eratosthenes.

2. It shows the working of the mind of Archimedes in
the discovery of mathematical truths, showing that he often
obtained his results by intuition or even by measurement,
rather than by an analytic form of reasoning, verifying
these results later by strict analysis.

3. It expresses definitely the fact that Archimedes was
the discoverer of those properties relating to the sphere
and cylinder that have been attributed to him and that are
given in his other works without a definite statement of
their authorship.

4. It shows that Archimedes was the first to state the
volume of the cylinder segment mentioned, and it gives
an interesting description of the mechanical method by
which he arrived at his result.

\Signature{David Eugene Smith.}{Teachers~College, Columbia~University.}
\PageSep{7}

\Section{Geometrical Solutions Derived from Mechanics.}

\noindent\textsc{Archimedes to Eratosthenes, Greeting:}

Some time ago I sent you some theorems I had discovered,
writing down only the propositions because I wished you to find
their demonstrations which had not been given. The propositions
of the theorems which I sent you were the following:

1. If in a perpendicular prism with a parallelogram\footnote
  {This must mean a square.}
for base
a cylinder is inscribed which has its bases in the opposite parallelograms\SameMark\
and its surface touching the other planes of the prism,
and if a plane is passed through the center of the circle that is the
base of the cylinder and one side of the square lying in the opposite
plane, then that plane will cut off from the cylinder a section which
is bounded by two planes, the intersecting plane and the one in
which the base of the cylinder lies, and also by as much of the
surface of the cylinder as lies between these same planes; and the
detached section of the cylinder is $\frac{1}{6}$~of the whole prism.

2. If in a cube a cylinder is inscribed whose bases lie in opposite
parallelograms\SameMark\ and whose surface touches the other four planes,
and if in the same cube a second cylinder is inscribed whose bases
lie in two other parallelograms\SameMark\ and whose surface touches the
four other planes, then the body enclosed by the surface of the
cylinder and comprehended within both cylinders will be equal to
$\frac{2}{3}$~of the whole cube.

These propositions differ essentially from those formerly discovered;
for then we compared those bodies (conoids, spheroids
and their segments) with the volume of cones and cylinders but none
of them was found to be equal to a body enclosed by planes. Each
of these bodies, on the other hand, which are enclosed by two planes
and cylindrical surfaces is found to be equal to a body enclosed
\PageSep{8}
by planes. The demonstration of these propositions I am accordingly
sending to you in this book.

Since I see, however, as I have previously said, that you are
a capable scholar and a prominent teacher of philosophy, and also
that you understand how to value a mathematical method of investigation
when the opportunity is offered, I have thought it well
to analyze and lay down for you in this same book a peculiar method
by means of which it will be possible for you to derive instruction
as to how certain mathematical questions may be investigated by
means of mechanics. And I am convinced that this is equally profitable
in demonstrating a proposition itself; for much that was made
evident to me through the medium of mechanics was later proved
by means of geometry because the treatment by the former method
had not yet been established by way of a demonstration. For of
course it is easier to establish a proof if one has in this way previously
obtained a conception of the questions, than for him to seek it
without such a preliminary notion. Thus in the familiar propositions
the demonstrations of which Eudoxos was the first to discover,
namely that a cone and a pyramid are one third the size of that
cylinder and prism respectively that have the same base and altitude,
no little credit is due to Democritos who was the first to make
that statement about these bodies without any demonstration. But
we are in a position to have found the present proposition in the
same way as the earlier one; and I have decided to write down and
make known the method partly because we have already talked
about it heretofore and so no one would think that we were spreading
abroad idle talk, and partly in the conviction that by this means
we are obtaining no slight advantage for mathematics, for indeed
I assume that some one among the investigators of to-day or in the
future will discover by the method here set forth still other propositions
which have not yet occurred to us.

In the first place we will now explain what was also first made
clear to us through mechanics, namely that a segment of a parabola
is $\frac{4}{3}$~of the triangle possessing the same base and equal altitude;
following which we will explain in order the particular propositions
discovered by the above mentioned method; and in the last part
of the book we will present the geometrical demonstrations of the
propositions.\footnote
  {In his ``Commentar,'' Professor Zeuthen calls attention to the fact that
  it was already known from Heron's recently discovered \Title{Metrica} that these
  propositions were contained in this treatise, and Professor Heiberg made the
  same comment in \Title{Hermes}.---Tr.}
\PageSep{9}

1. If one magnitude is taken away from another magnitude and
the same point is the center of gravity both of the whole and of the
part removed, then the same point is the center of gravity of the
remaining portion.

2. If one magnitude is taken away from another magnitude and
the center of gravity of the whole and of the part removed is not
the same point, the center of gravity of the remaining portion may
be found by prolonging the straight line which connects the centers
of gravity of the whole and of the part removed, and setting off
upon it another straight line which bears the same ratio to the
straight line between the aforesaid centers of gravity, as the weight
of the magnitude which has been taken away bears to the weight
of the one remaining [\Title{De plan.\ aequil.}\ I,~8].

3. If the centers of gravity of any number of magnitudes lie
upon the same straight line, then will the center of gravity of all the
magnitudes combined lie also upon the same straight line [Cf.\ \ibid\
I,~5].

4. The center of gravity of a straight line is the center of that
line [Cf.\ \ibid\ I,~4].

5. The center of gravity of a triangle is the point in which the
straight lines drawn from the angles of a triangle to the centers of
the opposite sides intersect [\Ibid\ I,~14].

6. The center of gravity of a parallelogram is the point where
its diagonals meet [\Ibid\ I,~10].

7. The center of gravity [of a circle] is the center [of that
circle].

8. The center of gravity of a cylinder [is the center of its axis].

9. The center of gravity of a prism is the center of its axis.

10. The center of gravity of a cone so divides its axis that the
section at the vertex is three times as great as the remainder.

11. Moreover together with the exercise here laid down I will
make use of the following proposition:

If any number of magnitudes stand in the same ratio to the
same number of other magnitudes which correspond pair by pair,
and if either all or some of the former magnitudes stand in any
ratio whatever to other magnitudes, and the latter in the same ratio
to the corresponding ones, then the sum of the magnitudes of the
first series will bear the same ratio to the sum of those taken from
the third series as the sum of those of the second series bears to
the sum of those taken from the fourth series [\Title{De Conoid.}~I].
\PageSep{10}


\Subsection{I.}

Let $\alpha\beta\gamma$ [\Fig{1}] be the segment of a parabola bounded by the
straight line~$\alpha\gamma$ and the parabola~$\alpha\beta\gamma$. Let $\alpha\gamma$~be bisected at~$\delta$, $\delta\beta\epsilon$~being
parallel to the diameter, and draw~$\alpha\beta$, and~$\beta\gamma$. Then the
segment~$\alpha\beta\gamma$ will be $\frac{4}{3}$~as great as the triangle~$\alpha\beta\gamma$.

From the points $\alpha$~and $\gamma$ draw $\alpha\zeta \| \delta\beta\epsilon$, and the tangent~$\gamma\zeta$;
produce [$\gamma\beta$~to~$\kappa$, and
make $\kappa\theta = \gamma\kappa$]. Think of~$\gamma\theta$
as a scale-beam with
the center at~$\kappa$ and let $\mu\xi$~be
any straight line whatever $\| \epsilon\delta$.
Now since $\gamma\beta\alpha$~is
a parabola, $\gamma\zeta$~a tangent
and $\gamma\delta$~an ordinate,
then $\epsilon\beta = \beta\delta$; for this indeed
has been proved in
the Elements [\ie, of
conic sections, cf.\  \Title{Quadr.\
parab.}~2]. For this reason
and because $\zeta\alpha$ and
$\mu\xi \| \epsilon\delta$, $\mu\nu = \nu\xi$, and $\zeta\kappa = \kappa\alpha$.
And because $\gamma\alpha : \alpha\xi = \mu\xi : \xi o$
(for this is shown
in a corollary, [cf.\ \Title{Quadr.\ parab.}~5]), $\gamma\alpha : \alpha\xi = \gamma\kappa : \kappa\nu$; and $\gamma\kappa = \kappa\theta$,
therefore $\theta\kappa : \kappa\nu = \mu\xi : \xi o$.  And because $\nu$~is the center of gravity of
the straight line~$\mu\xi$, since $\mu\nu = \nu\xi$, then if we make $\tau\eta = \xi o$ and $\theta$~as
its center of gravity so that $\tau\theta = \theta\eta$, the straight line~$\tau\theta\eta$ will be in
equilibrium with $\mu\xi$ in its present position because $\theta\nu$~is divided in
inverse proportion to the weights $\tau\eta$ and~$\mu\xi$, and $\theta\kappa : \kappa\nu = \mu\xi : \eta\tau$; therefore
$\kappa$~is the center of gravity of the combined weight of the two.
\Figure{1}{fig01}
In the same way all straight lines drawn in the triangle $\zeta\alpha\gamma \| \epsilon\delta$ are
in their present positions in equilibrium with their parts cut off by
the parabola, when these are transferred to~$\theta$, so that $\kappa$~is the center
of gravity of the combined weight of the two. And because the
triangle~$\gamma\zeta\alpha$ consists of the straight lines in the triangle~$\gamma\zeta\alpha$ and the
segment~$\alpha\beta\gamma$ consists of those straight lines within the segment of
the parabola corresponding to the straight line~$\xi o$, therefore the
triangle~$\zeta\alpha\gamma$ in its present position will be in equilibrium at the
point~$\kappa$ with the parabola-segment when this is transferred to~$\theta$ as
its center of gravity, so that $\kappa$~is the center of gravity of the combined
\PageSep{11}
weights of the two. Now let $\gamma\kappa$ be so divided at~$\chi$ that $\gamma\kappa = 3\kappa\chi$;
then $\chi$~will be the center of gravity of the triangle~$\alpha\zeta\gamma$, for this
has been shown in the Statics [cf.\ \Title{De plan.\ aequil.}\ I,~15, p.~186,~3
with Eutokios, S.~320,~5ff.]. Now the triangle~$\zeta\alpha\gamma$ in its present
position is in equilibrium at the point~$\kappa$ with the segment~$\beta\alpha\gamma$ when
this is transferred to~$\theta$ as its center of gravity, and the center of
gravity of the triangle~$\zeta\alpha\gamma$ is~$\chi$; hence $\Tri \alpha\zeta\gamma : \segm \alpha\beta\gamma$ when
transferred to~$\theta$ as its center of gravity $= \theta\kappa : \kappa\chi$. But $\theta\kappa = 3\kappa\chi$;
hence also $\Tri \alpha\zeta\gamma = 3 \segm \alpha\beta\gamma$.  But it is also true that $\Tri \zeta\alpha\gamma = 4\triangle\alpha\beta\gamma$
because $\zeta\kappa = \kappa\alpha$ and $\alpha\delta = \delta\gamma$; hence $\segm \alpha\beta\gamma = \frac{4}{3}$ the
$\Tri \alpha\beta\gamma$. This is of course clear.

It is true that this is not proved by what we have said here;
but it indicates that the result is correct. And so, as we have just
seen that it has not been proved but rather conjectured that the
result is correct we have devised a geometrical demonstration which
we made known some time ago and will again bring forward
farther on.


\Subsection{II.}

That a sphere is four times as large as a cone whose base is
equal to the largest circle of the sphere and whose altitude is equal
to the radius of the sphere, and that a cylinder whose base is equal
to the largest circle of the sphere and whose altitude is equal
to the diameter of the circle is one and a half times as large as the sphere,
may be seen by the present method in the following way:

\Figure{2}{fig02}

Let $\alpha\beta\gamma\delta$ [\Fig{2}] be the largest circle of a sphere and $\alpha\gamma$~and $\beta\delta$
its diameters perpendicular to each other; let there be in the sphere
a circle on the diameter~$\beta\delta$ perpendicular to the circle~$\alpha\beta\gamma\delta$, and
on this perpendicular circle let there be a cone erected with its
vertex at~$\alpha$; producing the convex surface of the cone, let it be
cut through~$\gamma$ by a plane parallel to its base; the result will be the
circle perpendicular to~$\alpha\gamma$ whose diameter will be~$\epsilon\zeta$. On this
circle erect a cylinder whose axis $= \alpha\gamma$ and whose vertical boundaries
are $\epsilon\lambda$ and~$\zeta\eta$. Produce $\gamma\alpha$ making $\alpha\theta = \gamma\alpha$ and think of~$\gamma\theta$ as
a scale-beam with its center at~$\alpha$. Then let $\mu\nu$~be any straight line
whatever drawn~$\| \beta\delta$ intersecting the circle~$\alpha\beta\gamma\delta$ in $\xi$~and~$o$, the
diameter~$\alpha\gamma$ in~$\sigma$, the straight line~$\alpha\epsilon$ in~$\pi$ and $\alpha\zeta$~in~$\rho$, and on the
straight line~$\mu\nu$ construct a plane perpendicular to~$\alpha\gamma$; it will intersect
the cylinder in a circle on the diameter~$\mu\nu$; the sphere~$\alpha\beta\gamma\delta$, in
a circle on the diameter~$\xi o$; the cone~$\alpha\epsilon\zeta$ in a circle on the diameter~$\pi\rho$.
\PageSep{12}
Now because $\gamma\alpha � \alpha\sigma = \mu\sigma � \sigma\pi$ (for $\alpha\gamma = \sigma\mu$, $\alpha\sigma = \pi\sigma$), and
$\gamma\alpha � \alpha\sigma = \alpha\xi^{2} = \xi\sigma^{2} + \alpha\pi^{2}$ then $\mu\sigma � \sigma\pi = \xi\sigma^{2} + \sigma\pi^{2}$. Moreover, because
$\gamma\alpha : \alpha\sigma = \mu\sigma : \sigma\pi$ and $\gamma\alpha = \alpha\theta$, therefore
$\theta\alpha : \alpha\sigma = \mu\sigma : \sigma\pi = \mu\sigma^{2} : \mu\sigma � \sigma\pi$.
But it has been
proved that $\xi\sigma^{2} + \sigma\pi^{2} = \mu\sigma � \sigma\pi$;
hence $\alpha\theta : \alpha\sigma = \mu\sigma^{2} : \xi\sigma^{2} + \sigma\pi^{2}$.
But it is true that
$\mu\sigma^{2} : \xi\sigma^{2} + \sigma\pi^{2} = \mu\nu^{2} : \xi\alpha^{2} + \pi\rho^{2} = {}$the
circle in the cylinder
whose diameter is~$\mu\nu :$
the circle in the cone
whose diameter is $\pi\rho + {}$the
circle in the sphere whose
diameter is~$\xi o$; hence $\theta\alpha : \alpha\sigma = {}$the
circle in the cylinder~$:$
the circle in the
sphere~$+$ the circle in the
cone. Therefore the circle in the cylinder in its present position
will be in equilibrium at the point~$\alpha$ with the two circles whose
diameters are $\xi o$ and~$\pi\rho$, if they are so transferred to~$\theta$ that $\theta$~is the
center of gravity of both. In the same way it can be shown that
when another straight line is drawn in the parallelogram $\xi\lambda \| \epsilon\zeta$,
and upon it a plane is erected perpendicular to~$\alpha\gamma$, the circle
produced in the cylinder in its present position will be in equilibrium
at the point~$\alpha$ with the two circles produced in the sphere and the
cone when they are transferred and so arranged on the scale-beam
at the point~$\theta$ that $\theta$~is the center of gravity of both. Therefore
if cylinder, sphere and cone are filled up with such circles then the
cylinder in its present position will be in equilibrium at the point~$\alpha$
with the sphere and the cone together, if they are transferred and
so arranged on the scale-beam at the point~$\theta$ that $\theta$~is the center of
gravity of both. Now since the bodies we have mentioned are in
equilibrium, the cylinder with $\kappa$~as its center of gravity, the sphere
and the cone transferred as we have said so that they have $\theta$~as
center of gravity, then $\theta\alpha : \alpha\kappa = \cylinder : \sphere + \cone$. But $\theta\alpha = 2\alpha\kappa$,
and hence also the $\cylinder = 2 � (\sphere + \cone)$. But it is also
true that the $\cylinder = 3 \cones$ [Euclid, \Title{Elem.}\ XII,~10], hence $3 \cones = 2 \cones + 2 \spheres$. If $2 \cones$ be subtracted from both
sides, then the cone whose axes form the triangle $\alpha\epsilon\zeta = 2 \spheres$.
But the cone whose axes form the triangle $\alpha\epsilon\zeta = 8 \cones$ whose axes
\PageSep{13}
form the triangle~$\alpha\beta\delta$ because $\epsilon\zeta = 2\beta\delta$, hence the aforesaid $8 \cones = 2 \spheres$.
Consequently the sphere whose greatest circle is~$\alpha\beta\gamma\delta$
is four times as large as the cone with its vertex at~$\alpha$, and whose
base is the circle on the diameter~$\beta\delta$ perpendicular to~$\alpha\gamma$.

Draw the straight lines~$\phi\beta\chi$ and $\psi\delta\omega \| \alpha\gamma$ through $\beta$~and~$\delta$ in
the parallelogram~$\lambda\zeta$ and imagine a cylinder whose bases are the
circles on the diameters $\phi\psi$ and~$\chi\omega$ and whose axis is~$\alpha\gamma$. Now
since the cylinder whose axes form the parallelogram~$\phi\omega$ is twice
as large as the cylinder whose axes form the parallelogram~$\phi\delta$ and
the latter is three times as large as the cone the triangle of whose
axes is~$\alpha\beta\delta$, as is shown in the Elements [Euclid, \Title{Elem.}\ XII,~10], the
cylinder whose axes form the parallelogram~$\phi\omega$ is six times as large
as the cone whose axes form the triangle~$\alpha\beta\delta$. But it was shown
that the sphere whose largest circle is~$\alpha\beta\gamma\delta$ is four times as large
as the same cone, consequently the cylinder is one and one half
times as large as the sphere,~\QED

After I had thus perceived that a sphere is four times as large
as the cone whose base is the largest circle of the sphere and whose
altitude is equal to its radius, it occurred to me that the surface of
a sphere is four times as great as its largest circle, in which I proceeded
from the idea that just as a circle is equal to a triangle whose
base is the periphery of the circle and whose altitude is equal to
its radius, so a sphere is equal to a cone whose base is the same as
the surface of the sphere and whose altitude is equal to the radius
of the sphere.


\Subsection{III.}

By this method it may also be seen that a cylinder whose base
is equal to the largest circle of a spheroid and whose altitude is
equal to the axis of the spheroid, is one and one half times as large
as the spheroid, and when this is recognized it becomes clear that
if a spheroid is cut through its center by a plane perpendicular to
its axis, one-half of the spheroid is twice as great as the cone whose
base is that of the segment and its axis the same.

For let a spheroid be cut by a plane through its axis and let
there be in its surface an ellipse~$\alpha\beta\gamma\delta$ [\Fig{3}] whose diameters are
$\alpha\gamma$ and~$\beta\delta$ and whose center is~$\kappa$ and let there be a circle in the
spheroid on the diameter~$\beta\delta$ perpendicular to~$\alpha\gamma$; then imagine a
cone whose base is the same circle but whose vertex is at~$\alpha$, and
producing its surface, let the cone be cut by a plane through~$\gamma$
\PageSep{14}
parallel to the base; the intersection will be a circle perpendicular
to~$\alpha\gamma$ with $\epsilon\zeta$~as its diameter. Now imagine a cylinder whose base
is the same circle with the diameter~$\epsilon\zeta$ and whose axis is~$\alpha\gamma$; let $\gamma\alpha$~be
produced so that $\alpha\theta = \gamma\alpha$; think of $\theta\gamma$ as a scale-beam with its
center at~$\alpha$ and in the parallelogram~$\lambda\theta$ draw a straight line $\mu\nu \| \epsilon\zeta$,
and on~$\mu\nu$ construct a plane perpendicular to~$\alpha\gamma$; this will intersect
the cylinder in a circle whose diameter is~$\mu\nu$, the spheroid in a circle
whose diameter is~$\xi o$ and the cone in a circle whose diameter is~$\pi\rho$.
Because $\gamma\alpha : \alpha\sigma = \epsilon\alpha : \alpha\pi = \mu\sigma : \sigma\pi$, and $\gamma\alpha = \alpha\theta$, therefore
$\theta\alpha : \alpha\sigma = \mu\sigma : \sigma\pi$.
But $\mu\sigma : \sigma\pi = \mu\sigma^{2} : \mu\sigma � \sigma\pi$ and $\mu\sigma � \sigma\pi = \pi\sigma^{2} + \sigma\xi^{2}$, for
$ \alpha\sigma � \sigma\gamma : \sigma\xi^{2} = \alpha\kappa � \kappa\gamma : \kappa\beta^{2} = \alpha\kappa^{2} : \kappa\beta^{2}$ (for both ratios are equal to the ratio
between the diameter and the
parameter [Apollonius, \Title{Con.}\ I,~21]) $= \alpha\sigma^{2} : \sigma\pi^{2}$ therefore
$\alpha\sigma^{2} : \alpha\sigma � \sigma\gamma = \pi\sigma^{2} : \sigma\xi^{2} = \sigma\pi^{2} : \sigma\pi � \pi\mu$,
consequently $\mu\pi � \pi\sigma = \sigma\xi^{2}$.
If $\pi\sigma^{2}$~is added to both
sides then $\mu\sigma � \sigma\pi = \pi\sigma^{2} + \sigma\xi^{2}$.
Therefore $\theta\alpha : \alpha\sigma = \mu\sigma^{2} : \pi\sigma^{2} + \sigma\xi^{2}$.
But $\mu\sigma^{2} : \sigma\xi^{2} + \sigma\pi^{2} = {}$the
circle in the cylinder whose
diameter is $\mu\nu :$~the circle with
the diameter $\xi o + {}$the circle
with the diameter~$\pi\rho$; hence
the circle whose diameter is~$\mu\nu$
will in its present position
be in equilibrium at the point~$\alpha$
with the two circles whose
diameters are $\xi o$ and~$\pi\rho$ when they are transferred and so arranged
on the scale-beam at the point~$\alpha$ that $\theta$~is the center of gravity of
both; and $\theta$~is the center of gravity of the two circles combined
whose diameters are $\xi o$ and~$\pi\rho$ when their position is changed,
\Figure{3}{fig03}
hence $\theta\alpha : \alpha\sigma = {}$the circle with the diameter~$\mu\nu :$ the two circles whose
diameters are $\xi o$ and~$\pi\rho$. In the same way it can be shown that
if another straight line is drawn in the parallelogram $\lambda\zeta \| \epsilon\zeta$ and on
this line last drawn a plane is constructed perpendicular to~$\alpha\gamma$, then
likewise the circle produced in the cylinder will in its present position
be in equilibrium at the point~$\alpha$ with the two circles combined
which have been produced in the spheroid and in the cone respectively
when they are so transferred to the point~$\theta$ on the scale-beam
that $\theta$~is the center of gravity of both. Then if cylinder, spheroid
\PageSep{15}
and cone are filled with such circles, the cylinder in its present position
will be in equilibrium at the point~$\alpha$ with the spheroid~$+$ the
cone if they are transferred and so arranged on the scale-beam at
the point~$\alpha$ that $\theta$~is the center of gravity of both. Now $\kappa$~is the
center of gravity of the cylinder, but $\theta$, as has been said, is the
center of gravity of the spheroid and cone together. Therefore
$\theta\alpha : \alpha\kappa = \cylinder : \spheroid + \cone$. But
$\alpha\theta = 2\alpha\kappa$, hence also the $\cylinder = 2 � (\spheroid + \cone) = 2 � \spheroid + 2 � \cone$. But the
$\cylinder = 3 � \cone$, hence $3 � \cone = 2 � \cone + 2 � \spheroid$. Subtract
$2 � \cone$ from both sides; then a cone whose axes form the triangle
$\alpha\epsilon\zeta = 2 � \spheroid$. But the same $\cone = 8$~cones whose axes form
the~$\triangle\alpha\beta\delta$; hence $8$~such cones $= 2 � \spheroid$, $4 � \cone = \spheroid$;
whence it follows that a spheroid is four times as great as a cone
whose vertex is at~$\alpha$, and whose base is the circle on the diameter~$\beta\delta$
perpendicular to~$\lambda\epsilon$, and one-half the spheroid is twice as great
as the same cone.

In the parallelogram~$\lambda\zeta$ draw the straight lines $\phi\chi$ and $\psi\omega \| \alpha\gamma$
through the points $\beta$~and~$\delta$ and imagine a cylinder whose bases
are the circles on the diameters $\phi\psi$ and~$\chi\omega$, and whose axis is~$\alpha\gamma$.
Now since the cylinder whose axes form the parallelogram~$\phi\omega$ is
twice as great as the cylinder whose axes form the parallelogram~$\phi\delta$
because their bases are equal but the axis of the first is twice as
great as the axis of the second, and since the cylinder whose axes
form the parallelogram~$\phi\delta$ is three times as great as the cone whose
vertex is at~$\alpha$ and whose base is the circle on the diameter~$\beta\delta$ perpendicular
to~$\alpha\gamma$, then the cylinder whose axes form the parallelogram~$\phi\omega$
is six times as great as the aforesaid cone. But it has
been shown that the spheroid is four times as great as the same
cone, hence the cylinder is one and one half times as great as the
spheroid.~\QED


\Subsection{IV.}

That a segment of a right conoid cut by a plane perpendicular
to its axis is one and one half times as great as the cone having
the same base and axis as the segment, can be proved by the same
method in the following way:

Let a right conoid be cut through its axis by a plane intersecting
the surface in a parabola~$\alpha\beta\gamma$ [\Fig{4}]; let it be also cut
by another plane perpendicular to the axis, and let their common
line of intersection be~$\beta\gamma$. Let the axis of the segment be~$\delta\alpha$ and
\PageSep{16}
let it be produced to~$\theta$ so that $\theta\alpha = \alpha\delta$. Now imagine $\delta\theta$ to be a
scale-beam with its center at~$\alpha$; let the base of the segment be
the circle on the diameter~$\beta\gamma$ perpendicular to~$\alpha\delta$; imagine a cone whose
base is the circle on the diameter~$\beta\gamma$, and whose vertex is at~$\alpha$.
Imagine also a cylinder whose base is the circle on the diameter~$\beta\gamma$
and its axis~$\alpha\delta$, and in the parallelogram let a straight line~$\mu\nu$ be
drawn $\| \beta\gamma$ and on~$\mu\nu$ construct a plane perpendicular to~$\alpha\delta$; it will
intersect the cylinder in a circle whose diameter is~$\mu\nu$, and the segment
of the right conoid in a circle whose diameter is~$\xi o$. Now
since $\beta\alpha\gamma$~is a parabola, $\alpha\delta$~its diameter and $\xi\sigma$~and $\beta\delta$ its ordinates,
then [\Title{Quadr.\ parab.}~3] $\delta\alpha : \alpha\sigma = \beta\delta^{2} : \xi\sigma^{2}$.  But $\delta\alpha = \alpha\theta$, therefore
$\theta\alpha : \alpha\sigma = \mu\sigma^{2} : \sigma\xi^{2}$. But $\mu\sigma^{2} : \sigma\xi^{2} = {}$the circle in the cylinder whose
diameter is~$\mu\nu :$ the circle in the segment of the right conoid whose
diameter is~$\xi o$, hence $\theta\alpha : \alpha\sigma = {}$the
circle with the diameter~$\mu\nu :$ the
circle with the diameter~$\xi o$; therefore
the circle in the cylinder
\Figure{4}{fig04}
whose diameter is~$\mu\nu$ is in its
present position, in equilibrium
at the point~$\alpha$ with the circle
whose diameter is~$\xi o$ if this be
transferred and so arranged on
the scale-beam at~$\theta$ that $\theta$~is its
center of gravity. And the center
of gravity of the circle whose
diameter is~$\mu\nu$ is at~$\sigma$, that of the
circle whose diameter is~$\xi o$ when
its position is changed, is~$\theta$, and we have the inverse proportion,
$\theta\alpha : \alpha\sigma = {}$the circle with the diameter~$\mu\nu :$ the circle with the diameter~$\xi o$.
In the same way it can be shown that if another straight line
be drawn in the parallelogram~$\epsilon\gamma \| \beta\gamma$ the circle formed in the
cylinder, will in its present position be in equilibrium at the point~$\alpha$
with that formed in the segment of the right conoid if the latter
is so transferred to~$\theta$ on the scale-beam that $\theta$~is its center of gravity.
Therefore if the cylinder and the segment of the right conoid
are filled up then the cylinder in its present position will be in
equilibrium at the point~$\alpha$ with the segment of the right conoid if
the latter is transferred and so arranged on the scale-beam at~$\theta$ that
$\theta$~is its center of gravity. And since these magnitudes are in equilibrium
at~$\alpha$, and $\kappa$~is the center of gravity of the cylinder, if $\alpha\delta$~is
bisected at~$\kappa$ and $\theta$~is the center of gravity of the segment transferred
\PageSep{17}
to that point, then we have the inverse proportion $\theta\alpha : \alpha\kappa = \cylinder : \segment$.
But $\theta\alpha = 2\alpha\kappa$ and also the $\cylinder = 2 � \segment$.
But the same cylinder is $3$~times as great as the cone whose base is
the circle on the diameter~$\beta\gamma$ and whose vertex is at~$\alpha$; therefore it
is clear that the segment is one and one half times as great as the
same cone.


\Subsection{V.}

That the center of gravity of a segment of a right conoid which
is cut off by a plane perpendicular to the axis, lies on the straight
line which is the axis of the segment divided in such a way that
the portion at the vertex is twice as great as the remainder, may
be perceived by our method in
the following way:

Let a segment of a right
conoid cut off by a plane perpendicular
to the axis be cut by
another plane through the axis,
and let the intersection in its surface
be the parabola~$\alpha\beta\gamma$ [\Fig{5}]
and let the common line of intersection
of the plane which cut off
the segment and of the intersecting
plane be~$\beta\gamma$; let the axis of
the segment and the diameter of
the parabola~$\alpha\beta\gamma$ be~$\alpha\delta$; produce~$\delta\alpha$
so that $\alpha\theta = \alpha\delta$ and imagine $\delta\theta$~to
be a scale-beam with its center
\Figure{5}{fig05}
at~$\alpha$; then inscribe a cone in the segment with the lateral boundaries
$\beta\alpha$ and $\alpha\gamma$ and in the parabola draw a straight line $\xi o \| \beta\gamma$ and let
it cut the parabola in $\xi$~and~$o$ and the lateral boundaries of the cone
in $\pi$~and~$\rho$.  Now because $\xi\sigma$~and $\beta\delta$ are drawn perpendicular to the
diameter of the parabola, $\delta\alpha : \alpha\sigma = \beta\delta^{2} : \xi\sigma^{2}$ [\Title{Quadr.\ parab.}~3]. But
$\delta\alpha : \alpha\sigma = \beta\delta : \pi\sigma = \beta\delta^{2} : \beta\delta � \pi\sigma$, therefore also $\beta\delta^{2} : \xi\sigma^{2} = \beta\delta^{2} : \beta\delta � \pi\sigma$.
Consequently $\xi\sigma^{2} = \beta\delta � \pi\sigma$ and $\beta\delta : \xi\sigma = \xi\sigma : \pi\sigma$, therefore $\beta\delta : \pi\sigma = \xi\sigma^{2} : \sigma\pi^{2}$.
But $\beta\delta : \pi\sigma = \delta\alpha : \alpha\sigma = \theta\alpha : \alpha\sigma$, therefore also $\theta\alpha : \alpha\sigma = \xi\sigma^{2} : \sigma\pi^{2}$.
On~$\xi o$ construct a plane perpendicular to~$\alpha\delta$; this will intersect the
segment of the right conoid in a circle whose diameter is~$\xi o$ and the
cone in a circle whose diameter is~$\pi\rho$. Now because $\theta\alpha : \alpha\sigma = \xi\sigma^{2} : \sigma\pi^{2}$
and $\xi\sigma^{2} : \sigma\pi^{2} = {}$the circle with the diameter~$\xi o :$ the circle with the
\PageSep{18}
diameter~$\pi\rho$, therefore $\theta\alpha : \alpha\sigma = {}$the circle whose diameter is~$\xi o :$ the circle
whose diameter is~$\pi\rho$. Therefore the circle whose diameter is~$\xi o$
will in its present position be in equilibrium at the point~$\alpha$ with the
circle whose diameter is~$\pi\rho$ when this is so transferred to~$\theta$ on the
scale-beam that $\theta$~is its center of gravity. Now since $\sigma$~is the center
of gravity of the circle whose diameter is~$\xi o$ in its present position,
and $\theta$~is the center of gravity of the circle whose diameter is~$\pi\rho$
if its position is changed as we have said, and inversely $\theta\alpha : \alpha\sigma = {}$the
circle with the diameter~$\xi o :$ the circle with the diameter~$\pi\rho$, then
the circles are in equilibrium at the point~$\alpha$.  In the same way it
can be shown that if another straight line is drawn in the parabola
$\| \beta\gamma$ and on this line last drawn a plane is constructed perpendicular
to~$\alpha\delta$, the circle formed in the segment of the right conoid will in
its present position be in equilibrium at the point~$\alpha$ with the circle
formed in the cone, if the latter is transferred and so arranged on
the scale-beam at~$\theta$ that $\theta$~is its center of gravity.  Therefore if the
segment and the cone are filled up with circles, all circles in the
segment will be in their present positions in equilibrium at the point~$\alpha$
with all circles of the cone if the latter are transferred and so arranged
on the scale-beam at the point~$\theta$ that $\theta$~is their center of
gravity. Therefore also the segment of the right conoid in its
present position will be in equilibrium at the point~$\alpha$ with the cone if
it is transferred and so arranged on the scale-beam at~$\theta$ that $\theta$~is its
center of gravity. Now because the center of gravity of both magnitudes
taken together is~$\alpha$, but that of the cone alone when its
position is changed is~$\theta$, then the center of gravity of the remaining
magnitude lies on~$\alpha\theta$ extended towards~$\alpha$ if $\alpha\kappa$~is cut off in such a
way that $\alpha\theta : \alpha\kappa = \segment : \cone$. But the segment is one
and one
half the size of the cone, consequently $\alpha\theta = \frac{3}{2}\alpha\kappa$ and $\kappa$,~the center of
gravity of the right conoid, so divides~$\alpha\delta$ that the portion at the
vertex of the segment is twice as large as the remainder.


\Subsection{VI.}

[The center of gravity of a hemisphere is so divided on its
axis] that the portion near the surface of the hemisphere is in the
ratio of $5 : 3$ to the remaining portion.

Let a sphere be cut by a plane through its center intersecting
the surface in the circle~$\alpha\beta\gamma\delta$ [\Fig{6}], $\alpha\gamma$~and $\beta\delta$ being two diameters
of the circle perpendicular to each other. Let a plane be constructed
\PageSep{19}
on~$\beta\delta$ perpendicular to~$\alpha\gamma$. Then imagine a cone whose base
is the circle with the diameter~$\beta\delta$, whose vertex is at~$\alpha$ and its
lateral boundaries are $\beta\alpha$ and~$\alpha\delta$; let $\gamma\alpha$~be produced so that $\alpha\theta = \gamma\alpha$,
imagine the straight line~$\theta\gamma$ to be a scale-beam with its center at~$\alpha$
and in the \Chg{semi-circle}{semicircle}~$\beta\alpha\delta$ draw a straight line $\xi o \| \beta\delta$; let it cut
the circumference of the semicircle in $\xi$~and~$o$, the lateral boundaries
of the cone in $\pi$~and~$\rho$, and $\alpha\gamma$~in~$\epsilon$.  On~$\xi o$ construct a plane perpendicular
to~$\alpha\epsilon$; it will intersect the hemisphere in a circle with the
diameter~$\xi o$, and the cone in a circle with the diameter~$\pi\rho$. Now
because $\alpha\gamma : \alpha\epsilon = \xi\alpha^{2} : \alpha\epsilon^{2}$ and $\xi\alpha^{2} = \alpha\epsilon^{2} + \epsilon\xi^{2}$ and $\alpha\epsilon = \epsilon\pi$, therefore $\alpha\gamma : \alpha\epsilon = \xi\epsilon^{2} + \epsilon\pi^{2} : \epsilon\pi^{2}$.
But $\xi\epsilon^{2} + \epsilon\pi^{2} : \epsilon\pi^{2} = {}$the circle with the diameter $\xi o + {}$
the circle with the diameter~$\pi\rho :$ the circle with the diameter~$\pi\rho$, and
$\gamma\alpha = \alpha\theta$, hence $\theta\alpha : \alpha\epsilon = {}$the circle with the diameter $\xi o + {}$the circle with
\Figure[0.5\textwidth]{6}{fig06}
the diameter $\pi\rho :$ circle with the diameter~$\pi\rho$.
Therefore the two circles whose diameters
are $\xi o$ and $\pi\rho$ in their present position are in
equilibrium at the point~$\alpha$ with the circle
whose diameter is~$\pi\rho$ if it is transferred and
so arranged at~$\theta$ that $\theta$~is its center of gravity.
Now since the center of gravity of the two
circles whose diameters are $\xi o$ and $\pi\rho$ in their
present position [is the point~$\epsilon$, but of the
circle whose diameter is~$\pi\rho$ when its position
is changed is the point~$\theta$, then $\theta\alpha : \alpha\epsilon = {}$the
circles whose diameters are]~$\xi o$[,~$\pi\rho :$ the
circle whose diameter is~$\pi\rho$.  In the same
way if another straight line in the] hemisphere~$\beta\alpha\delta$
[is drawn $\| \beta\delta$ and a plane is
constructed] perpendicular to~[$\alpha\gamma$ the] two
[circles produced in the cone and in the hemisphere
are in their position] in equilibrium at~$\alpha$ [with the circle
which is produced in the cone] if it is transferred and arranged on
the scale at~$\theta$. [Now if] the hemisphere and the cone [are filled
up with circles then all circles in the] hemisphere and those [in the
cone] will in their present position be in equilibrium [with all
circles] in the cone, if these are transferred and so arranged on the
scale-beam at~$\theta$ that $\theta$~is their center of gravity; [therefore the
hemisphere and cone also] are in their position [in equilibrium at
the point~$\alpha$] with the cone if it is transferred and so arranged [on
the scale-beam at~$\theta$] that $\theta$~is its center of gravity.
\PageSep{20}


\Subsection{VII.}

By [this method] it may also be perceived that [any segment
whatever] of a sphere bears the same ratio to a cone having the
same [base] and axis [that the radius of the sphere~$+$ the axis of the
opposite segment~$:$ the axis of the opposite segment]\dotfill \\
and [\Fig{7}] on~$\mu\nu$ construct a plane perpendicular to~$\alpha\gamma$; it will
intersect the cylinder in a circle whose diameter is~$\mu\nu$, the segment
\Figure{7}{fig07}
of the sphere in a circle whose diameter is~$\xi o$ and the cone whose
base is the circle on the diameter~$\epsilon\zeta$ and whose vertex is at~$\alpha$ in
a circle whose diameter
is~$\pi\rho$. In the same way
as before it may be
shown that a circle whose
diameter is~$\mu\nu$ is in its
present position in equilibrium
at~$\alpha$ with the two
%[** TN: Opening bracket has no matching close bracket in the original]
circles [whose diameters
are $\xi o$ and $\pi\rho$ if they are
so arranged on the scale-beam
that $\theta$~is their center
of gravity. [And the
same can be proved of
all corresponding circles.]
Now since cylinder,
cone, and spherical
segment are filled up
with such circles, the
cylinder in its present
position [will be in equilibrium at~$\alpha$] with the cone~$+$ the spherical
segment if they are transferred and attached to the scale-beam at~$\theta$.
Divide~$\alpha\eta$ at $\phi$~and $\chi$ so that $\alpha\chi = \chi\eta$ and $\eta\phi = \frac{1}{3}\alpha\phi$; then $\chi$~will be the
center of gravity of the cylinder because it is the center of the axis~$\alpha\eta$.
Now because the above mentioned bodies are in equilibrium
at~$\alpha$, $\cylinder : \cone$ with the diameter of its base $\epsilon\zeta + {}$the spherical
segment $\beta\alpha\delta = \theta\alpha : \alpha\chi$. And because $\eta\alpha = 3\eta\phi$ then [$\gamma\eta � \eta\phi$] $= \frac{1}{3}\alpha\eta � \eta\gamma$.
Therefore also $\gamma\eta � \eta\phi = \frac{1}{3}\beta\eta^{2}$.~\dotfill


\Subsection{VIIa.}

In the same way it may be perceived that any segment of an
ellipsoid cut off by a perpendicular plane, bears the same ratio to
\PageSep{21}
a cone having the same base and the same axis, as half of the axis
of the ellipsoid~$+$ the axis of the opposite segment bears to the axis
of the opposite segment.~\dotfill

\Subsection{VIII.}

\noindent\dotfill\\
produce $\alpha\gamma$ [\Fig{8}] making $\alpha\theta = \alpha\gamma$ and $\gamma\xi = {}$the radius of the sphere;
imagine $\gamma\theta$~to be a scale-beam with a center at~$\alpha$, and in the plane
\Figure[0.5\textwidth]{8}{fig08}
cutting off the segment inscribe a circle with its center at~$\eta$ and its
radius $= \alpha\eta$; on this circle construct a cone with its vertex at~$\alpha$ and
its lateral boundaries $\alpha\epsilon$ and~$\alpha\zeta$. Then draw a straight line $\kappa\lambda \| \epsilon\zeta$;
let it cut the circumference of the
segment at $\kappa$~and~$\lambda$, the lateral boundaries
of the $\cone \alpha\epsilon\zeta$ at $\rho$~and~$o$ and $\alpha\gamma$
at~$\pi$. Now because $\alpha\gamma : \alpha\pi = \alpha\kappa^{2} : \alpha\pi^{2}$
and $\kappa\alpha^{2} = \alpha\pi^{2} + \pi\kappa^{2}$ and $\alpha\pi^{2} = \pi o^{2}$ (since
also $\alpha\eta^{2} = \epsilon\eta^{2}$), then $\gamma\alpha : \alpha\pi = \kappa\pi^{2} + \pi o^{2} : o\pi^{2}$.
But $\kappa\pi^{2} + \pi o^{2} : \pi o^{2} = {}$the circle
with the diameter $\kappa\lambda + {}$the circle with
the diameter $o\rho :$ the circle with the diameter~$o\rho$
and $\gamma\alpha = \alpha\theta$; therefore
$\theta\alpha : \alpha\pi = {}$the circle with the diameter
$\kappa\lambda + {}$the circle with the diameter $o\rho :$
the circle with the diameter~$o\rho$. Now
since the circle with the diameter $\kappa\lambda + {}$
the circle with the diameter $o\rho :$ the
circle with the diameter $o\rho = \alpha\theta : \pi\alpha$,
let the circle with the diameter~$o\rho$ be
transferred and so arranged on the
scale-beam at~$\theta$ that $\theta$~is its center of
gravity; then $\theta\alpha : \alpha\pi = {}$the circle with
the diameter $\kappa\lambda + {}$the circle with the diameter~$o\rho$ in their present
positions~$:$ the circle with the diameter~$o\rho$ if it is transferred and
so arranged on the scale-beam at~$\theta$ that $\theta$~is its center of gravity.
Therefore the circles in the $\segment \beta\alpha\delta$ and in the $\cone \alpha\epsilon\zeta$ are in
equilibrium at~$\alpha$ with that in the $\cone \alpha\epsilon\zeta$. And in the same way
all circles in the $\segment \beta\alpha\delta$ and in the $\cone \alpha\epsilon\zeta$ in their present
positions are in equilibrium at the point~$\alpha$ with all circles in the
$\cone \alpha\epsilon\zeta$ if they are transferred and so arranged on the scale-beam
at~$\theta$ that $\theta$~is their center of gravity; then also the spherical $\segment \alpha\beta\delta$
\PageSep{22}
and the $\cone \alpha\epsilon\zeta$ in their present positions are in equilibrium
at the point~$\alpha$ with the $\cone \epsilon\alpha\zeta$ if it is transferred and so arranged
on the scale-beam at~$\theta$ that $\theta$~is its center of gravity. Let the $\cylinder \mu\nu$
equal the cone whose base is the circle with the diameter~$\epsilon\zeta$
and whose vertex is at~$\alpha$ and let $\alpha\eta$~be so divided at~$\phi$ that $\alpha\eta = 4\phi\eta$;
then $\phi$~is the center of gravity of the $\cone \epsilon\alpha\zeta$ as has been previously
proved. Moreover let the $\cylinder \mu\nu$ be so cut by a perpendicularly
intersecting plane that the $\cylinder \mu$ is in equilibrium with the
$\cone \epsilon\alpha\zeta$. Now since the $\segment \alpha\beta\delta + {}$the $\cone \epsilon\alpha\zeta$ in their present
positions are in equilibrium at~$\alpha$ with the $\cone \epsilon\alpha\zeta$ if it is transferred
and so arranged on the scale-beam at~$\theta$ that $\theta$~is its center
of gravity, and $\cylinder \mu\nu = \cone \epsilon\alpha\zeta$ and the two cylinders $\mu + \nu$
are moved to~$\theta$ and $\mu\nu$~is in equilibrium with both bodies, then will
also the $\cylinder \nu$ be in equilibrium with the segment of the sphere
at the point~$\alpha$. And since the spherical $\segment \beta\alpha\delta :$ the cone whose
base is the circle with the diameter~$\beta\delta$, and whose vertex is at $\alpha = \xi\eta : \eta\gamma$
(for this has previously been proved [\Title{De sph.\ et cyl.}\ II,~2
Coroll.])\ and $\cone \beta\alpha\delta : \cone \epsilon\alpha\zeta = {}$the circle with the diameter
$\beta\delta :$ the circle with the diameter $\epsilon\zeta = \beta\eta^{2} : \eta\epsilon^{2}$, and $\beta\eta^{2} = \gamma\eta � \eta\alpha$,
$\eta\epsilon^{2} = \eta\alpha^{2}$, and $\gamma\eta � \eta\alpha : \eta\alpha^{2} = \gamma\eta : \eta\alpha$, therefore $\cone \beta\alpha\delta : \cone \epsilon\alpha\zeta = \gamma\eta : \eta\alpha$.
But we have shown that $\cone \beta\alpha\delta : \segment \beta\alpha\delta = \gamma\eta : \eta\xi$,
hence {\selectlanguage{greek}di' \~isou} $\segment \beta\alpha\delta : \cone \epsilon\alpha\zeta = \xi\eta : \eta\alpha$. And because $\alpha\chi : \chi\eta = \eta\alpha + 4\eta\gamma : \alpha\eta + 2\eta\gamma$
so inversely $\eta\chi : \chi\alpha = 2\gamma\eta + \eta\alpha : 4\gamma\eta + \eta\alpha$ and by addition
$\eta\alpha : \alpha\chi = 6\gamma\eta + 2\eta\alpha : \eta\alpha + 4\eta\gamma$. But $\eta\xi = \frac{1}{4} (6\eta\gamma + 2\eta\alpha)$ and $\gamma\phi = \frac{1}{4} (4\eta\gamma + \eta\alpha)$;
for that is evident. Hence $\eta\alpha : \alpha\chi = \xi\eta : \gamma\phi$, consequently
also $\xi\eta : \eta\alpha = \gamma\phi : \chi\alpha$. But it was also demonstrated that $\xi\eta : \eta\alpha = {}$the
segment whose vertex is at~$\alpha$ and whose base is the circle with the
diameter $\beta\delta :$ the cone whose vertex is at~$\alpha$ and whose base is the circle with the
diameter~$\epsilon\zeta$; hence $\segment \beta\alpha\delta : \cone \epsilon\alpha\zeta = \gamma\phi : \chi\alpha$.
And since the $\cylinder \mu$ is in equilibrium with the $\cone \epsilon\alpha\zeta$ at~$\alpha$, and $\theta$~is
the center of gravity of the cylinder while $\phi$~is that of the $\cone \epsilon\alpha\zeta$,
then $\cone \epsilon\alpha\zeta : \cylinder \mu = \theta\alpha : \alpha\phi = \gamma\alpha : \alpha\phi$. But $\cylinder \mu\nu = \cone \epsilon\alpha\zeta$;
hence by subtraction, $\cylinder \mu : \cylinder \nu = \alpha\phi : \gamma\phi$. And
$\cylinder \mu\nu = \cone \epsilon\alpha\zeta$; hence $\cone \epsilon\alpha\zeta : \cylinder \nu = \gamma\alpha : \gamma\phi = \theta\alpha : \gamma\phi$.
But it was also demonstrated that $\segment \beta\alpha\delta : \cone \epsilon\alpha\zeta = \gamma\phi : \chi\alpha$;
hence {\selectlanguage{greek}di' \~isou} $\segment \beta\alpha\delta : \cylinder \nu = \zeta\alpha : \alpha\chi$. And it was demonstrated
that $\segment \beta\alpha\delta$ is in equilibrium at~$\alpha$ with the cylinder~$\nu$
and $\theta$~is the center of gravity of the $\cylinder \nu$, consequently the
point~$\chi$ is also the center of gravity of the $\segment \beta\alpha\delta$.


\Subsection{IX.}

In a similar way it can also be perceived that the center of gravity
of any segment of an ellipsoid lies on the straight line which is
the axis of the segment so divided that the portion at the vertex
of the segment bears the same ratio to the remaining portion as the
axis of the segment${} + 4$~times the axis of the opposite segment
bears to the axis of the segment~$+$ twice the axis of the opposite
segment.


\Subsection{X.}

It can also be seen by this method that [a segment of a hyperboloid]
bears the same ratio to a cone having the same base and axis
as the segment, that the axis of the segment${} + 3$~times the addition
to the axis bears to the axis of the segment of the hyperboloid~$+$ twice
its addition [\Title{De Conoid.}~25]; and that the center of gravity of the
hyperboloid so divides the axis that the part at the vertex bears the
same ratio to the rest that three times the axis~$+$ eight times the
addition to the axis bears to the axis of the hyperboloid${} + 4$~times
the addition to the axis, and many other points which I will leave
aside since the method has been made clear by the examples already
given and only the demonstrations of the above given theorems remain
to be stated.


\Subsection{XI.}

When in a perpendicular prism with square bases a cylinder is
inscribed whose bases lie in opposite squares and whose curved
surface touches the four other parallelograms, and when a plane is
passed through the center of the circle which is the base of the
cylinder and one side of the opposite square, then the body which
is cut off by this plane [from the cylinder] will be $\frac{1}{6}$~of the entire
prism. This can be perceived through the present method and
when it is so warranted we will pass over to the geometrical proof
of it.

Imagine a perpendicular prism with square bases and a cylinder
inscribed in the prism in the way we have described. Let the
prism be cut through the axis by a plane perpendicular to the plane
which cuts off the section of the cylinder; this will intersect the
prism containing the cylinder in the parallelogram~$\alpha\beta$ [\Fig{9}] and
the common intersecting line of the plane which cuts off the section
of the cylinder and the plane lying through the axis perpendicular
\PageSep{24}
to the one cutting off the section of the cylinder will be~$\beta\gamma$; let the
axis of the cylinder and the prism be~$\gamma\delta$ which is bisected at right
angles by~$\epsilon\zeta$ and on~$\epsilon\zeta$ let a plane be constructed perpendicular to~$\gamma\delta$.
This will intersect the prism in a square and the cylinder in a
circle.

\Figures{9}{fig09}{10}{fig10}

Now let the intersection of the prism be the square~$\mu\nu$ [\Fig{10}],
that of the cylinder, the circle~$\xi o\pi\rho$ and let the circle touch the sides
of the square at the points $\xi$,~$o$,~$\pi$ and~$\rho$; let the common line of
intersection of the plane cutting off the cylinder-section and that
passing through~$\epsilon\zeta$ perpendicular to the axis of the cylinder, be~$\kappa\lambda$;
this line is bisected by~$\pi\theta\xi$. In the semicircle~$o\pi\rho$ draw a straight
line~$\sigma\tau$ perpendicular to~$\pi\chi$, on~$\sigma\tau$ construct a plane perpendicular
to~$\xi\pi$ and produce it to both sides of the plane enclosing the circle~$\xi o\pi\rho$;
this will intersect the half-cylinder whose base is the semicircle~$o\pi\rho$
and whose altitude is the axis of the prism, in a parallelogram
one side of which $= \sigma\tau$ and the other~$=$ the vertical boundary
of the cylinder, and it will intersect the cylinder-section likewise
in a parallelogram of which one side is~$\sigma\tau$ and the other~$\mu\nu$ [\Fig{9}];
and accordingly $\mu\nu$~will be drawn in the parallelogram $\delta\epsilon \| \beta\omega$ and
will cut off $\epsilon\iota = \pi\chi$. Now because $\epsilon\gamma$~is a parallelogram and $\nu\iota \| \theta\gamma$,
and $\epsilon\theta$~and $\beta\gamma$ cut the parallels, therefore $\epsilon\theta : \theta\iota = \omega\gamma : \gamma\nu = \beta\omega : \upsilon\nu$. But
$\beta\omega : \upsilon\nu = $ parallelogram in the half-cylinder~$:$ parallelogram in the
cylinder-section, therefore both parallelograms have the same side~$\sigma\tau$;
and $\epsilon\theta = \theta\pi$, $\iota\theta = \chi\theta$; and since $\pi\theta = \theta\xi$ therefore $\theta\xi : \theta\chi = $ parallelogram
in half-cylinder~$:$ parallelogram in the cylinder-section.
Imagine the parallelogram in the cylinder-section transferred and
so brought to~$\xi$ that $\xi$~is its center of gravity, and further imagine
\PageSep{25}
$\pi\xi$~to be a scale-beam with its center at~$\theta$; then the parallelogram in
the half-cylinder in its present position is in equilibrium at the
point~$\theta$ with the parallelogram in the cylinder-section when it is transferred
and so arranged on the scale-beam at~$\xi$ that $\xi$~is its center of
gravity. And since $\chi$~is the center of gravity in the parallelogram
in the half-cylinder, and $\xi$~that of the parallelogram in the cylinder-section
when its position is changed, and $\xi\theta : \theta\chi =$ the parallelogram
whose center of gravity is $\chi :$ the parallelogram whose center of
gravity is~$\xi$, then the parallelogram whose center of gravity is~$\chi$
will be in equilibrium at~$\theta$ with the parallelogram whose center of
gravity is~$\xi$. In this way it can be proved that if another straight
line is drawn in the semicircle~$o\pi\rho$ perpendicular to~$\pi\theta$ and on this
straight line a plane is constructed perpendicular to~$\pi\theta$ and is produced
towards both sides of the plane in which the circle~$\xi o\pi\rho$ lies,
then the parallelogram formed in the half-cylinder in its present
position will be in equilibrium at the point~$\theta$ with the parallelogram
formed in the cylinder-section if this is transferred and so arranged
on the scale-beam at~$\xi$ that $\xi$~is its center of gravity; therefore also
all parallelograms in the half-cylinder in their present positions will
be in equilibrium at the point~$\theta$ with all parallelograms of the
cylinder-section if they are transferred and attached to the scale-beam
at the point~$\xi$; consequently also the half-cylinder in its present
position will be in equilibrium at the point~$\theta$ with the cylinder-section
if it is transferred and so arranged on the scale-beam at~$\xi$
that $\xi$~is its center of gravity.


\Subsection{XII.}

Let the parallelogram~$\mu\nu$ be perpendicular to the axis [of the
circle]~$\xi o$ [$\pi\rho$] [\Fig{11}]. Draw $\theta\mu$~and
$\theta\eta$ and erect upon them two planes perpendicular
to the plane in which the
semicircle~$o\pi\rho$ lies and extend these
planes on both sides. The result is a
prism whose base is a triangle similar
to~$\theta\mu\eta$ and whose altitude is equal to
the axis of the cylinder, and this prism
is $\frac{1}{4}$~of the entire prism which contains
the cylinder. In the semicircle~$o\pi\rho$ and
in the square~$\mu\nu$ draw two straight lines
$\kappa\lambda$ and $\tau\upsilon$ at equal distances from~$\pi\xi$;
\Figure{11}{fig11}
these will cut the circumference of the semicircle~$o\pi\rho$ at the points
\PageSep{26}
$\kappa$~and~$\tau$, the diameter~$o\rho$ at $\sigma$~and~$\zeta$ and the straight lines $\theta\eta$ and $\theta\mu$
at $\phi$~and~$\chi$. Upon $\kappa\lambda$~and $\tau\upsilon$ construct two planes perpendicular
to~$o\rho$ and extend them towards both sides of the plane in which lies
the circle~$\xi o\pi\rho$; they will intersect the half-cylinder whose base is
the semicircle~$o\pi\rho$ and whose altitude is that of the cylinder, in a
parallelogram one side of which $= \kappa\sigma$ and the other~$=$ the axis of
the cylinder; and they will intersect the prism~$\theta\eta\mu$ likewise in a
parallelogram one side of which is equal to~$\lambda\chi$ and the other equal
to the axis, and in the same way the half-cylinder in a parallelogram
one side of which $= \tau\zeta$ and the other~$=$ the axis of the cylinder, and
the prism in a parallelogram one side of which $= \nu\phi$ and the other~$=$
the axis of the cylinder.~\dotfill


\Subsection{XIII.}

Let the square~$\alpha\beta\gamma\delta$ [\Fig{12}] be the base of a perpendicular
prism with square bases and let a cylinder be inscribed in the prism
whose base is the circle~$\epsilon\zeta\eta\theta$ which
touches the sides of the parallelogram~$\alpha\beta\gamma\delta$
at $\epsilon$,~$\zeta$,~$\eta$, and~$\theta$. Pass a plane
through its center and the side in the
square opposite the square~$\alpha\beta\gamma\delta$ corresponding
to the side~$\gamma\delta$; this will cut
off from the whole prism a second prism
which is $\frac{1}{4}$~the size of the whole prism
and which will be bounded by three
parallelograms and two opposite triangles.
In the semicircle~$\epsilon\zeta\eta$ describe
a parabola whose origin is~$\eta\epsilon$ and whose
axis is~$\zeta\kappa$, and in the parallelogram~$\delta\eta$ draw $\mu\nu \| \kappa\zeta$; this will cut
the circumference of the semicircle at~$\xi$, the parabola at~$\lambda$, and
$\mu\nu � \nu\lambda = \nu\zeta^{2}$ (for this is evident [Apollonios, \Title{Con.}\ I,~11]). Therefore
$\mu\nu : \nu\lambda = \kappa\eta^{2} : \lambda\sigma^{2}$. Upon~$\mu\nu$ construct a plane parallel to~$\epsilon\eta$; this will
intersect the prism cut off from the whole prism in a right-angled
triangle one side of which is~$\mu\nu$ and the other a straight line in the
plane upon~$\gamma\delta$ perpendicular to~$\gamma\delta$ at~$\nu$ and equal to the axis of the
cylinder, but whose hypotenuse is in the intersecting plane. It will
\Figure{12}{fig12}
intersect the portion which is cut off from the cylinder by the plane
passed through $\epsilon\eta$~and the side of the square opposite the side~$\gamma\delta$
in a right-angled triangle one side of which is~$\mu\xi$ and the other
a straight line drawn in the surface of the cylinder perpendicular
\PageSep{27}
to the plane~$\kappa\nu$, and the hypotenuse \dotfill \\
and all the triangles in the prism~$:$ all the triangles in the cylinder-section~$=$
all the straight lines in the parallelogram~$\delta\eta :$ all the straight
lines between the parabola and the straight line~$\epsilon\eta$. And the prism
consists of the triangles in the prism, the cylinder-section of those
in the cylinder-section, the parallelogram~$\delta\eta$ of the straight lines
in the parallelogram $\delta\eta \| \kappa\zeta$ and the segment of the parabola of the
straight lines cut off by the parabola and the straight line~$\epsilon\eta$; hence
prism~$:$ cylinder-section~$=$ parallelogram $\eta\delta : \segment \epsilon\zeta\eta$ that is
bounded by the parabola and the straight line~$\epsilon\eta$. But the parallelogram
$\delta\eta = \frac{3}{2}$~the segment bounded by the parabola and the straight
line~$\epsilon\eta$ as indeed has been shown in the previously published work,
hence also the prism is equal to one and one half times the cylinder-section.
Therefore when the cylinder-section $= 2$, the prism $= 3$ and
the whole prism containing the cylinder equals~$12$, because it is four
times the size of the other prism; hence the cylinder-section is equal
to $\frac{1}{6}$~of the prism,~\QED


\Subsection{XIV.}

[Inscribe a cylinder in] a perpendicular prism with square
bases [and let it be cut by a plane passed through the center of the
base of the cylinder and one side of the opposite square.] Then this
plane will cut off a prism from the whole prism and a portion of
the cylinder from the cylinder. It may be proved that the portion
cut off from the cylinder by the plane is one-sixth of the whole
prism. But first we will prove that it is possible to inscribe a solid
figure in the cylinder-section and to circumscribe another composed
of prisms of equal altitude and with similar triangles as bases, so
that the circumscribed figure exceeds the inscribed less than any
given magnitude.~\dotfill

But it has been shown that the prism cut off by the inclined plane
$< \frac{3}{2}$ the body inscribed in the cylinder-section. Now the prism
cut off by the inclined plane~$:$ the body inscribed in the cylinder-section~$=$
parallelogram $\delta\eta :$ the parallelograms which are inscribed
in the segment bounded by the parabola and the straight line~$\epsilon\eta$.
Hence the parallelogram $\delta\eta < \frac{3}{2}$~the parallelograms in the segment
bounded by the parabola and the straight line~$\epsilon\eta$. But this is impossible
because we have shown elsewhere that the parallelogram~$\delta\eta$
is one and one half times the segment bounded by the parabola
and the straight line~$\epsilon\eta$, consequently is~\dotfill \\
not greater~\dotfill\null

And all prisms in the prism cut off by the inclined plane~$:$ all
prisms in the figure described around the cylinder-section~$=$ all
parallelograms in the parallelogram $\delta\eta :$ all parallelograms
in the figure which is described around the segment bounded by the
parabola and the straight line~$\epsilon\eta$, \ie, the prism cut off by the inclined
plane~$:$ the figure described around the cylinder-section~$=$
parallelogram $\delta\eta :$ the figure bounded by the parabola and the
straight line~$\epsilon\eta$. But the prism cut off by the inclined plane is
greater than one and one half times the solid figure circumscribed
around the cylinder-section~\dotfill

\vfill

%%%%%%%%%%%%%%%%%%%%%%%%% GUTENBERG LICENSE %%%%%%%%%%%%%%%%%%%%%%%%%%
\BackMatter
\PGLicense
\begin{PGtext}
End of the Project Gutenberg EBook of Geometrical Solutions Derived from
Mechanics, by Archimedes

*** END OF THIS PROJECT GUTENBERG EBOOK GEOMETRICAL SOLUTIONS ***

***** This file should be named 7825-pdf.pdf or 7825-pdf.zip *****
This and all associated files of various formats will be found in:
        http://www.gutenberg.org/7/8/2/7825/

Produced by Gordon Keener

Updated editions will replace the previous one--the old editions
will be renamed.

Creating the works from public domain print editions means that no
one owns a United States copyright in these works, so the Foundation
(and you!) can copy and distribute it in the United States without
permission and without paying copyright royalties.  Special rules,
set forth in the General Terms of Use part of this license, apply to
copying and distributing Project Gutenberg-tm electronic works to
protect the PROJECT GUTENBERG-tm concept and trademark.  Project
Gutenberg is a registered trademark, and may not be used if you
charge for the eBooks, unless you receive specific permission.  If you
do not charge anything for copies of this eBook, complying with the
rules is very easy.  You may use this eBook for nearly any purpose
such as creation of derivative works, reports, performances and
research.  They may be modified and printed and given away--you may do
practically ANYTHING with public domain eBooks.  Redistribution is
subject to the trademark license, especially commercial
redistribution.



*** START: FULL LICENSE ***

THE FULL PROJECT GUTENBERG LICENSE
PLEASE READ THIS BEFORE YOU DISTRIBUTE OR USE THIS WORK

To protect the Project Gutenberg-tm mission of promoting the free
distribution of electronic works, by using or distributing this work
(or any other work associated in any way with the phrase "Project
Gutenberg"), you agree to comply with all the terms of the Full Project
Gutenberg-tm License available with this file or online at
  www.gutenberg.org/license.


Section 1.  General Terms of Use and Redistributing Project Gutenberg-tm
electronic works

1.A.  By reading or using any part of this Project Gutenberg-tm
electronic work, you indicate that you have read, understand, agree to
and accept all the terms of this license and intellectual property
(trademark/copyright) agreement.  If you do not agree to abide by all
the terms of this agreement, you must cease using and return or destroy
all copies of Project Gutenberg-tm electronic works in your possession.
If you paid a fee for obtaining a copy of or access to a Project
Gutenberg-tm electronic work and you do not agree to be bound by the
terms of this agreement, you may obtain a refund from the person or
entity to whom you paid the fee as set forth in paragraph 1.E.8.

1.B.  "Project Gutenberg" is a registered trademark.  It may only be
used on or associated in any way with an electronic work by people who
agree to be bound by the terms of this agreement.  There are a few
things that you can do with most Project Gutenberg-tm electronic works
even without complying with the full terms of this agreement.  See
paragraph 1.C below.  There are a lot of things you can do with Project
Gutenberg-tm electronic works if you follow the terms of this agreement
and help preserve free future access to Project Gutenberg-tm electronic
works.  See paragraph 1.E below.

1.C.  The Project Gutenberg Literary Archive Foundation ("the Foundation"
or PGLAF), owns a compilation copyright in the collection of Project
Gutenberg-tm electronic works.  Nearly all the individual works in the
collection are in the public domain in the United States.  If an
individual work is in the public domain in the United States and you are
located in the United States, we do not claim a right to prevent you from
copying, distributing, performing, displaying or creating derivative
works based on the work as long as all references to Project Gutenberg
are removed.  Of course, we hope that you will support the Project
Gutenberg-tm mission of promoting free access to electronic works by
freely sharing Project Gutenberg-tm works in compliance with the terms of
this agreement for keeping the Project Gutenberg-tm name associated with
the work.  You can easily comply with the terms of this agreement by
keeping this work in the same format with its attached full Project
Gutenberg-tm License when you share it without charge with others.

1.D.  The copyright laws of the place where you are located also govern
what you can do with this work.  Copyright laws in most countries are in
a constant state of change.  If you are outside the United States, check
the laws of your country in addition to the terms of this agreement
before downloading, copying, displaying, performing, distributing or
creating derivative works based on this work or any other Project
Gutenberg-tm work.  The Foundation makes no representations concerning
the copyright status of any work in any country outside the United
States.

1.E.  Unless you have removed all references to Project Gutenberg:

1.E.1.  The following sentence, with active links to, or other immediate
access to, the full Project Gutenberg-tm License must appear prominently
whenever any copy of a Project Gutenberg-tm work (any work on which the
phrase "Project Gutenberg" appears, or with which the phrase "Project
Gutenberg" is associated) is accessed, displayed, performed, viewed,
copied or distributed:

This eBook is for the use of anyone anywhere at no cost and with
almost no restrictions whatsoever.  You may copy it, give it away or
re-use it under the terms of the Project Gutenberg License included
with this eBook or online at www.gutenberg.org

1.E.2.  If an individual Project Gutenberg-tm electronic work is derived
from the public domain (does not contain a notice indicating that it is
posted with permission of the copyright holder), the work can be copied
and distributed to anyone in the United States without paying any fees
or charges.  If you are redistributing or providing access to a work
with the phrase "Project Gutenberg" associated with or appearing on the
work, you must comply either with the requirements of paragraphs 1.E.1
through 1.E.7 or obtain permission for the use of the work and the
Project Gutenberg-tm trademark as set forth in paragraphs 1.E.8 or
1.E.9.

1.E.3.  If an individual Project Gutenberg-tm electronic work is posted
with the permission of the copyright holder, your use and distribution
must comply with both paragraphs 1.E.1 through 1.E.7 and any additional
terms imposed by the copyright holder.  Additional terms will be linked
to the Project Gutenberg-tm License for all works posted with the
permission of the copyright holder found at the beginning of this work.

1.E.4.  Do not unlink or detach or remove the full Project Gutenberg-tm
License terms from this work, or any files containing a part of this
work or any other work associated with Project Gutenberg-tm.

1.E.5.  Do not copy, display, perform, distribute or redistribute this
electronic work, or any part of this electronic work, without
prominently displaying the sentence set forth in paragraph 1.E.1 with
active links or immediate access to the full terms of the Project
Gutenberg-tm License.

1.E.6.  You may convert to and distribute this work in any binary,
compressed, marked up, nonproprietary or proprietary form, including any
word processing or hypertext form.  However, if you provide access to or
distribute copies of a Project Gutenberg-tm work in a format other than
"Plain Vanilla ASCII" or other format used in the official version
posted on the official Project Gutenberg-tm web site (www.gutenberg.org),
you must, at no additional cost, fee or expense to the user, provide a
copy, a means of exporting a copy, or a means of obtaining a copy upon
request, of the work in its original "Plain Vanilla ASCII" or other
form.  Any alternate format must include the full Project Gutenberg-tm
License as specified in paragraph 1.E.1.

1.E.7.  Do not charge a fee for access to, viewing, displaying,
performing, copying or distributing any Project Gutenberg-tm works
unless you comply with paragraph 1.E.8 or 1.E.9.

1.E.8.  You may charge a reasonable fee for copies of or providing
access to or distributing Project Gutenberg-tm electronic works provided
that

- You pay a royalty fee of 20% of the gross profits you derive from
     the use of Project Gutenberg-tm works calculated using the method
     you already use to calculate your applicable taxes.  The fee is
     owed to the owner of the Project Gutenberg-tm trademark, but he
     has agreed to donate royalties under this paragraph to the
     Project Gutenberg Literary Archive Foundation.  Royalty payments
     must be paid within 60 days following each date on which you
     prepare (or are legally required to prepare) your periodic tax
     returns.  Royalty payments should be clearly marked as such and
     sent to the Project Gutenberg Literary Archive Foundation at the
     address specified in Section 4, "Information about donations to
     the Project Gutenberg Literary Archive Foundation."

- You provide a full refund of any money paid by a user who notifies
     you in writing (or by e-mail) within 30 days of receipt that s/he
     does not agree to the terms of the full Project Gutenberg-tm
     License.  You must require such a user to return or
     destroy all copies of the works possessed in a physical medium
     and discontinue all use of and all access to other copies of
     Project Gutenberg-tm works.

- You provide, in accordance with paragraph 1.F.3, a full refund of any
     money paid for a work or a replacement copy, if a defect in the
     electronic work is discovered and reported to you within 90 days
     of receipt of the work.

- You comply with all other terms of this agreement for free
     distribution of Project Gutenberg-tm works.

1.E.9.  If you wish to charge a fee or distribute a Project Gutenberg-tm
electronic work or group of works on different terms than are set
forth in this agreement, you must obtain permission in writing from
both the Project Gutenberg Literary Archive Foundation and Michael
Hart, the owner of the Project Gutenberg-tm trademark.  Contact the
Foundation as set forth in Section 3 below.

1.F.

1.F.1.  Project Gutenberg volunteers and employees expend considerable
effort to identify, do copyright research on, transcribe and proofread
public domain works in creating the Project Gutenberg-tm
collection.  Despite these efforts, Project Gutenberg-tm electronic
works, and the medium on which they may be stored, may contain
"Defects," such as, but not limited to, incomplete, inaccurate or
corrupt data, transcription errors, a copyright or other intellectual
property infringement, a defective or damaged disk or other medium, a
computer virus, or computer codes that damage or cannot be read by
your equipment.

1.F.2.  LIMITED WARRANTY, DISCLAIMER OF DAMAGES - Except for the "Right
of Replacement or Refund" described in paragraph 1.F.3, the Project
Gutenberg Literary Archive Foundation, the owner of the Project
Gutenberg-tm trademark, and any other party distributing a Project
Gutenberg-tm electronic work under this agreement, disclaim all
liability to you for damages, costs and expenses, including legal
fees.  YOU AGREE THAT YOU HAVE NO REMEDIES FOR NEGLIGENCE, STRICT
LIABILITY, BREACH OF WARRANTY OR BREACH OF CONTRACT EXCEPT THOSE
PROVIDED IN PARAGRAPH 1.F.3.  YOU AGREE THAT THE FOUNDATION, THE
TRADEMARK OWNER, AND ANY DISTRIBUTOR UNDER THIS AGREEMENT WILL NOT BE
LIABLE TO YOU FOR ACTUAL, DIRECT, INDIRECT, CONSEQUENTIAL, PUNITIVE OR
INCIDENTAL DAMAGES EVEN IF YOU GIVE NOTICE OF THE POSSIBILITY OF SUCH
DAMAGE.

1.F.3.  LIMITED RIGHT OF REPLACEMENT OR REFUND - If you discover a
defect in this electronic work within 90 days of receiving it, you can
receive a refund of the money (if any) you paid for it by sending a
written explanation to the person you received the work from.  If you
received the work on a physical medium, you must return the medium with
your written explanation.  The person or entity that provided you with
the defective work may elect to provide a replacement copy in lieu of a
refund.  If you received the work electronically, the person or entity
providing it to you may choose to give you a second opportunity to
receive the work electronically in lieu of a refund.  If the second copy
is also defective, you may demand a refund in writing without further
opportunities to fix the problem.

1.F.4.  Except for the limited right of replacement or refund set forth
in paragraph 1.F.3, this work is provided to you 'AS-IS', WITH NO OTHER
WARRANTIES OF ANY KIND, EXPRESS OR IMPLIED, INCLUDING BUT NOT LIMITED TO
WARRANTIES OF MERCHANTABILITY OR FITNESS FOR ANY PURPOSE.

1.F.5.  Some states do not allow disclaimers of certain implied
warranties or the exclusion or limitation of certain types of damages.
If any disclaimer or limitation set forth in this agreement violates the
law of the state applicable to this agreement, the agreement shall be
interpreted to make the maximum disclaimer or limitation permitted by
the applicable state law.  The invalidity or unenforceability of any
provision of this agreement shall not void the remaining provisions.

1.F.6.  INDEMNITY - You agree to indemnify and hold the Foundation, the
trademark owner, any agent or employee of the Foundation, anyone
providing copies of Project Gutenberg-tm electronic works in accordance
with this agreement, and any volunteers associated with the production,
promotion and distribution of Project Gutenberg-tm electronic works,
harmless from all liability, costs and expenses, including legal fees,
that arise directly or indirectly from any of the following which you do
or cause to occur: (a) distribution of this or any Project Gutenberg-tm
work, (b) alteration, modification, or additions or deletions to any
Project Gutenberg-tm work, and (c) any Defect you cause.


Section  2.  Information about the Mission of Project Gutenberg-tm

Project Gutenberg-tm is synonymous with the free distribution of
electronic works in formats readable by the widest variety of computers
including obsolete, old, middle-aged and new computers.  It exists
because of the efforts of hundreds of volunteers and donations from
people in all walks of life.

Volunteers and financial support to provide volunteers with the
assistance they need are critical to reaching Project Gutenberg-tm's
goals and ensuring that the Project Gutenberg-tm collection will
remain freely available for generations to come.  In 2001, the Project
Gutenberg Literary Archive Foundation was created to provide a secure
and permanent future for Project Gutenberg-tm and future generations.
To learn more about the Project Gutenberg Literary Archive Foundation
and how your efforts and donations can help, see Sections 3 and 4
and the Foundation information page at www.gutenberg.org


Section 3.  Information about the Project Gutenberg Literary Archive
Foundation

The Project Gutenberg Literary Archive Foundation is a non profit
501(c)(3) educational corporation organized under the laws of the
state of Mississippi and granted tax exempt status by the Internal
Revenue Service.  The Foundation's EIN or federal tax identification
number is 64-6221541.  Contributions to the Project Gutenberg
Literary Archive Foundation are tax deductible to the full extent
permitted by U.S. federal laws and your state's laws.

The Foundation's principal office is located at 4557 Melan Dr. S.
Fairbanks, AK, 99712., but its volunteers and employees are scattered
throughout numerous locations.  Its business office is located at 809
North 1500 West, Salt Lake City, UT 84116, (801) 596-1887.  Email
contact links and up to date contact information can be found at the
Foundation's web site and official page at www.gutenberg.org/contact

For additional contact information:
     Dr. Gregory B. Newby
     Chief Executive and Director
     gbnewby@pglaf.org

Section 4.  Information about Donations to the Project Gutenberg
Literary Archive Foundation

Project Gutenberg-tm depends upon and cannot survive without wide
spread public support and donations to carry out its mission of
increasing the number of public domain and licensed works that can be
freely distributed in machine readable form accessible by the widest
array of equipment including outdated equipment.  Many small donations
($1 to $5,000) are particularly important to maintaining tax exempt
status with the IRS.

The Foundation is committed to complying with the laws regulating
charities and charitable donations in all 50 states of the United
States.  Compliance requirements are not uniform and it takes a
considerable effort, much paperwork and many fees to meet and keep up
with these requirements.  We do not solicit donations in locations
where we have not received written confirmation of compliance.  To
SEND DONATIONS or determine the status of compliance for any
particular state visit www.gutenberg.org/donate

While we cannot and do not solicit contributions from states where we
have not met the solicitation requirements, we know of no prohibition
against accepting unsolicited donations from donors in such states who
approach us with offers to donate.

International donations are gratefully accepted, but we cannot make
any statements concerning tax treatment of donations received from
outside the United States.  U.S. laws alone swamp our small staff.

Please check the Project Gutenberg Web pages for current donation
methods and addresses.  Donations are accepted in a number of other
ways including checks, online payments and credit card donations.
To donate, please visit:  www.gutenberg.org/donate


Section 5.  General Information About Project Gutenberg-tm electronic
works.

Professor Michael S. Hart was the originator of the Project Gutenberg-tm
concept of a library of electronic works that could be freely shared
with anyone.  For forty years, he produced and distributed Project
Gutenberg-tm eBooks with only a loose network of volunteer support.

Project Gutenberg-tm eBooks are often created from several printed
editions, all of which are confirmed as Public Domain in the U.S.
unless a copyright notice is included.  Thus, we do not necessarily
keep eBooks in compliance with any particular paper edition.

Most people start at our Web site which has the main PG search facility:

     www.gutenberg.org

This Web site includes information about Project Gutenberg-tm,
including how to make donations to the Project Gutenberg Literary
Archive Foundation, how to help produce our new eBooks, and how to
subscribe to our email newsletter to hear about new eBooks.
\end{PGtext}

% %%%%%%%%%%%%%%%%%%%%%%%%%%%%%%%%%%%%%%%%%%%%%%%%%%%%%%%%%%%%%%%%%%%%%%% %
%                                                                         %
% End of the Project Gutenberg EBook of Geometrical Solutions Derived from%
% Mechanics, by Archimedes                                                %
%                                                                         %
% *** END OF THIS PROJECT GUTENBERG EBOOK GEOMETRICAL SOLUTIONS ***       %
%                                                                         %
% ***** This file should be named 7825-t.tex or 7825-t.zip *****          %
% This and all associated files of various formats will be found in:      %
%         http://www.gutenberg.org/7/8/2/7825/                            %
%                                                                         %
% %%%%%%%%%%%%%%%%%%%%%%%%%%%%%%%%%%%%%%%%%%%%%%%%%%%%%%%%%%%%%%%%%%%%%%% %

\end{document}
###
@ControlwordReplace = (
  ['\\ie', 'i.e.'],
  ['\\ibid', 'ibid.'],
  ['\\Ibid', 'Ibid.'],
  ['\\QED', 'Q.E.D.']
  );

@ControlwordArguments = (
  ['\\BookMark', 1, 0, '', '', 1, 0, '', ''],
  ['\\Signature', 1, 1, '', ' ', 1, 1, '', ''],
  ['\\Figure', 0, 0, '', '', 1, 0, '', '', 1, 0, '<GRAPHIC>', ''],
  ['\\Figures', 1, 0, '', '', 1, 0, '', '', 1, 0, '', '', 1, 0, '<GRAPHIC>', ''],
  ['\\Fig', 1, 1, 'Fig. ', ''],
  ['\\First', 1, 1, '', ''],
  ['\\Section', 1, 1, '', ''],
  ['\\Subsection', 1, 1, '', ''],
  ['\\Chg', 1, 0, '', '', 1, 1, '', '']
  );
$PageSeparator = qr/^\\PageSep/;
$CustomClean = 'print "\\nCustom cleaning in progress...";
my $cline = 0;
 while ($cline <= $#file) {
   $file[$cline] =~ s/--------[^\n]*\n//; # strip page separators
   $cline++
 }
 print "done\\n";';
###
This is pdfTeX, Version 3.1415926-2.5-1.40.14 (TeX Live 2013/Debian) (format=pdflatex 2014.9.6)  1 NOV 2014 12:57
entering extended mode
 %&-line parsing enabled.
**7825-t.tex
(./7825-t.tex
LaTeX2e <2011/06/27>
Babel <3.9h> and hyphenation patterns for 78 languages loaded.
(/usr/share/texlive/texmf-dist/tex/latex/base/book.cls
Document Class: book 2007/10/19 v1.4h Standard LaTeX document class
(/usr/share/texlive/texmf-dist/tex/latex/base/leqno.clo
File: leqno.clo 1998/08/17 v1.1c Standard LaTeX option (left equation numbers)
) (/usr/share/texlive/texmf-dist/tex/latex/base/bk12.clo
File: bk12.clo 2007/10/19 v1.4h Standard LaTeX file (size option)
)
\c@part=\count79
\c@chapter=\count80
\c@section=\count81
\c@subsection=\count82
\c@subsubsection=\count83
\c@paragraph=\count84
\c@subparagraph=\count85
\c@figure=\count86
\c@table=\count87
\abovecaptionskip=\skip41
\belowcaptionskip=\skip42
\bibindent=\dimen102
) (/usr/share/texlive/texmf-dist/tex/latex/base/inputenc.sty
Package: inputenc 2008/03/30 v1.1d Input encoding file
\inpenc@prehook=\toks14
\inpenc@posthook=\toks15
(/usr/share/texlive/texmf-dist/tex/latex/base/latin1.def
File: latin1.def 2008/03/30 v1.1d Input encoding file
)) (/usr/share/texlive/texmf-dist/tex/latex/base/ifthen.sty
Package: ifthen 2001/05/26 v1.1c Standard LaTeX ifthen package (DPC)
) (/usr/share/texlive/texmf-dist/tex/generic/babel/babel.sty
Package: babel 2013/12/03 3.9h The Babel package
(/usr/share/texlive/texmf-dist/tex/generic/babel-greek/greek.ldf
Language: greek 2013/12/03 v1.8a Greek support for the babel system
(/usr/share/texlive/texmf-dist/tex/generic/babel/babel.def
File: babel.def 2013/12/03 3.9h Babel common definitions
\babel@savecnt=\count88
\U@D=\dimen103
) (/usr/share/texlive/texmf-dist/tex/latex/greek-fontenc/lgrenc.def
File: lgrenc.def 2013/07/16 v0.9 LGR Greek font encoding definitions
(/usr/share/texlive/texmf-dist/tex/latex/greek-fontenc/greek-fontenc.def
File: greek-fontenc.def 2013/11/28 v0.11 Common Greek font encoding definitions

))) (/usr/share/texlive/texmf-dist/tex/generic/babel-english/english.ldf
Language: english 2012/08/20 v3.3p English support from the babel system
\l@canadian = a dialect from \language\l@american 
\l@australian = a dialect from \language\l@british 
\l@newzealand = a dialect from \language\l@british 
)) (/usr/share/texlive/texmf-dist/tex/latex/amsmath/amsmath.sty
Package: amsmath 2013/01/14 v2.14 AMS math features
\@mathmargin=\skip43
For additional information on amsmath, use the `?' option.
(/usr/share/texlive/texmf-dist/tex/latex/amsmath/amstext.sty
Package: amstext 2000/06/29 v2.01
(/usr/share/texlive/texmf-dist/tex/latex/amsmath/amsgen.sty
File: amsgen.sty 1999/11/30 v2.0
\@emptytoks=\toks16
\ex@=\dimen104
)) (/usr/share/texlive/texmf-dist/tex/latex/amsmath/amsbsy.sty
Package: amsbsy 1999/11/29 v1.2d
\pmbraise@=\dimen105
) (/usr/share/texlive/texmf-dist/tex/latex/amsmath/amsopn.sty
Package: amsopn 1999/12/14 v2.01 operator names
)
\inf@bad=\count89
LaTeX Info: Redefining \frac on input line 210.
\uproot@=\count90
\leftroot@=\count91
LaTeX Info: Redefining \overline on input line 306.
\classnum@=\count92
\DOTSCASE@=\count93
LaTeX Info: Redefining \ldots on input line 378.
LaTeX Info: Redefining \dots on input line 381.
LaTeX Info: Redefining \cdots on input line 466.
\Mathstrutbox@=\box26
\strutbox@=\box27
\big@size=\dimen106
LaTeX Font Info:    Redeclaring font encoding OML on input line 566.
LaTeX Font Info:    Redeclaring font encoding OMS on input line 567.
\macc@depth=\count94
\c@MaxMatrixCols=\count95
\dotsspace@=\muskip10
\c@parentequation=\count96
\dspbrk@lvl=\count97
\tag@help=\toks17
\row@=\count98
\column@=\count99
\maxfields@=\count100
\andhelp@=\toks18
\eqnshift@=\dimen107
\alignsep@=\dimen108
\tagshift@=\dimen109
\tagwidth@=\dimen110
\totwidth@=\dimen111
\lineht@=\dimen112
\@envbody=\toks19
\multlinegap=\skip44
\multlinetaggap=\skip45
\mathdisplay@stack=\toks20
LaTeX Info: Redefining \[ on input line 2665.
LaTeX Info: Redefining \] on input line 2666.
) (/usr/share/texlive/texmf-dist/tex/latex/amsfonts/amssymb.sty
Package: amssymb 2013/01/14 v3.01 AMS font symbols
(/usr/share/texlive/texmf-dist/tex/latex/amsfonts/amsfonts.sty
Package: amsfonts 2013/01/14 v3.01 Basic AMSFonts support
\symAMSa=\mathgroup4
\symAMSb=\mathgroup5
LaTeX Font Info:    Overwriting math alphabet `\mathfrak' in version `bold'
(Font)                  U/euf/m/n --> U/euf/b/n on input line 106.
)) (/usr/share/texlive/texmf-dist/tex/latex/base/alltt.sty
Package: alltt 1997/06/16 v2.0g defines alltt environment
) (/usr/share/texlive/texmf-dist/tex/latex/tools/indentfirst.sty
Package: indentfirst 1995/11/23 v1.03 Indent first paragraph (DPC)
) (/usr/share/texlive/texmf-dist/tex/latex/graphics/graphicx.sty
Package: graphicx 1999/02/16 v1.0f Enhanced LaTeX Graphics (DPC,SPQR)
(/usr/share/texlive/texmf-dist/tex/latex/graphics/keyval.sty
Package: keyval 1999/03/16 v1.13 key=value parser (DPC)
\KV@toks@=\toks21
) (/usr/share/texlive/texmf-dist/tex/latex/graphics/graphics.sty
Package: graphics 2009/02/05 v1.0o Standard LaTeX Graphics (DPC,SPQR)
(/usr/share/texlive/texmf-dist/tex/latex/graphics/trig.sty
Package: trig 1999/03/16 v1.09 sin cos tan (DPC)
) (/usr/share/texlive/texmf-dist/tex/latex/latexconfig/graphics.cfg
File: graphics.cfg 2010/04/23 v1.9 graphics configuration of TeX Live
)
Package graphics Info: Driver file: pdftex.def on input line 91.
(/usr/share/texlive/texmf-dist/tex/latex/pdftex-def/pdftex.def
File: pdftex.def 2011/05/27 v0.06d Graphics/color for pdfTeX
(/usr/share/texlive/texmf-dist/tex/generic/oberdiek/infwarerr.sty
Package: infwarerr 2010/04/08 v1.3 Providing info/warning/error messages (HO)
) (/usr/share/texlive/texmf-dist/tex/generic/oberdiek/ltxcmds.sty
Package: ltxcmds 2011/11/09 v1.22 LaTeX kernel commands for general use (HO)
)
\Gread@gobject=\count101
))
\Gin@req@height=\dimen113
\Gin@req@width=\dimen114
) (/usr/share/texlive/texmf-dist/tex/latex/tools/calc.sty
Package: calc 2007/08/22 v4.3 Infix arithmetic (KKT,FJ)
\calc@Acount=\count102
\calc@Bcount=\count103
\calc@Adimen=\dimen115
\calc@Bdimen=\dimen116
\calc@Askip=\skip46
\calc@Bskip=\skip47
LaTeX Info: Redefining \setlength on input line 76.
LaTeX Info: Redefining \addtolength on input line 77.
\calc@Ccount=\count104
\calc@Cskip=\skip48
) (/usr/share/texlive/texmf-dist/tex/latex/fancyhdr/fancyhdr.sty
\fancy@headwidth=\skip49
\f@ncyO@elh=\skip50
\f@ncyO@erh=\skip51
\f@ncyO@olh=\skip52
\f@ncyO@orh=\skip53
\f@ncyO@elf=\skip54
\f@ncyO@erf=\skip55
\f@ncyO@olf=\skip56
\f@ncyO@orf=\skip57
) (/usr/share/texlive/texmf-dist/tex/latex/geometry/geometry.sty
Package: geometry 2010/09/12 v5.6 Page Geometry
(/usr/share/texlive/texmf-dist/tex/generic/oberdiek/ifpdf.sty
Package: ifpdf 2011/01/30 v2.3 Provides the ifpdf switch (HO)
Package ifpdf Info: pdfTeX in PDF mode is detected.
) (/usr/share/texlive/texmf-dist/tex/generic/oberdiek/ifvtex.sty
Package: ifvtex 2010/03/01 v1.5 Detect VTeX and its facilities (HO)
Package ifvtex Info: VTeX not detected.
) (/usr/share/texlive/texmf-dist/tex/generic/ifxetex/ifxetex.sty
Package: ifxetex 2010/09/12 v0.6 Provides ifxetex conditional
)
\Gm@cnth=\count105
\Gm@cntv=\count106
\c@Gm@tempcnt=\count107
\Gm@bindingoffset=\dimen117
\Gm@wd@mp=\dimen118
\Gm@odd@mp=\dimen119
\Gm@even@mp=\dimen120
\Gm@layoutwidth=\dimen121
\Gm@layoutheight=\dimen122
\Gm@layouthoffset=\dimen123
\Gm@layoutvoffset=\dimen124
\Gm@dimlist=\toks22
) (/usr/share/texlive/texmf-dist/tex/latex/hyperref/hyperref.sty
Package: hyperref 2012/11/06 v6.83m Hypertext links for LaTeX
(/usr/share/texlive/texmf-dist/tex/generic/oberdiek/hobsub-hyperref.sty
Package: hobsub-hyperref 2012/05/28 v1.13 Bundle oberdiek, subset hyperref (HO)

(/usr/share/texlive/texmf-dist/tex/generic/oberdiek/hobsub-generic.sty
Package: hobsub-generic 2012/05/28 v1.13 Bundle oberdiek, subset generic (HO)
Package: hobsub 2012/05/28 v1.13 Construct package bundles (HO)
Package hobsub Info: Skipping package `infwarerr' (already loaded).
Package hobsub Info: Skipping package `ltxcmds' (already loaded).
Package: ifluatex 2010/03/01 v1.3 Provides the ifluatex switch (HO)
Package ifluatex Info: LuaTeX not detected.
Package hobsub Info: Skipping package `ifvtex' (already loaded).
Package: intcalc 2007/09/27 v1.1 Expandable calculations with integers (HO)
Package hobsub Info: Skipping package `ifpdf' (already loaded).
Package: etexcmds 2011/02/16 v1.5 Avoid name clashes with e-TeX commands (HO)
Package etexcmds Info: Could not find \expanded.
(etexcmds)             That can mean that you are not using pdfTeX 1.50 or
(etexcmds)             that some package has redefined \expanded.
(etexcmds)             In the latter case, load this package earlier.
Package: kvsetkeys 2012/04/25 v1.16 Key value parser (HO)
Package: kvdefinekeys 2011/04/07 v1.3 Define keys (HO)
Package: pdftexcmds 2011/11/29 v0.20 Utility functions of pdfTeX for LuaTeX (HO
)
Package pdftexcmds Info: LuaTeX not detected.
Package pdftexcmds Info: \pdf@primitive is available.
Package pdftexcmds Info: \pdf@ifprimitive is available.
Package pdftexcmds Info: \pdfdraftmode found.
Package: pdfescape 2011/11/25 v1.13 Implements pdfTeX's escape features (HO)
Package: bigintcalc 2012/04/08 v1.3 Expandable calculations on big integers (HO
)
Package: bitset 2011/01/30 v1.1 Handle bit-vector datatype (HO)
Package: uniquecounter 2011/01/30 v1.2 Provide unlimited unique counter (HO)
)
Package hobsub Info: Skipping package `hobsub' (already loaded).
Package: letltxmacro 2010/09/02 v1.4 Let assignment for LaTeX macros (HO)
Package: hopatch 2012/05/28 v1.2 Wrapper for package hooks (HO)
Package: xcolor-patch 2011/01/30 xcolor patch
Package: atveryend 2011/06/30 v1.8 Hooks at the very end of document (HO)
Package atveryend Info: \enddocument detected (standard20110627).
Package: atbegshi 2011/10/05 v1.16 At begin shipout hook (HO)
Package: refcount 2011/10/16 v3.4 Data extraction from label references (HO)
Package: hycolor 2011/01/30 v1.7 Color options for hyperref/bookmark (HO)
) (/usr/share/texlive/texmf-dist/tex/latex/oberdiek/auxhook.sty
Package: auxhook 2011/03/04 v1.3 Hooks for auxiliary files (HO)
) (/usr/share/texlive/texmf-dist/tex/latex/oberdiek/kvoptions.sty
Package: kvoptions 2011/06/30 v3.11 Key value format for package options (HO)
)
\@linkdim=\dimen125
\Hy@linkcounter=\count108
\Hy@pagecounter=\count109
(/usr/share/texlive/texmf-dist/tex/latex/hyperref/pd1enc.def
File: pd1enc.def 2012/11/06 v6.83m Hyperref: PDFDocEncoding definition (HO)
)
\Hy@SavedSpaceFactor=\count110
(/usr/share/texlive/texmf-dist/tex/latex/latexconfig/hyperref.cfg
File: hyperref.cfg 2002/06/06 v1.2 hyperref configuration of TeXLive
)
Package hyperref Info: Option `hyperfootnotes' set `false' on input line 4319.
Package hyperref Info: Option `bookmarks' set `true' on input line 4319.
Package hyperref Info: Option `linktocpage' set `false' on input line 4319.
Package hyperref Info: Option `pdfdisplaydoctitle' set `true' on input line 431
9.
Package hyperref Info: Option `pdfpagelabels' set `true' on input line 4319.
Package hyperref Info: Option `bookmarksopen' set `true' on input line 4319.
Package hyperref Info: Option `colorlinks' set `true' on input line 4319.
Package hyperref Info: Hyper figures OFF on input line 4443.
Package hyperref Info: Link nesting OFF on input line 4448.
Package hyperref Info: Hyper index ON on input line 4451.
Package hyperref Info: Plain pages OFF on input line 4458.
Package hyperref Info: Backreferencing OFF on input line 4463.
Package hyperref Info: Implicit mode ON; LaTeX internals redefined.
Package hyperref Info: Bookmarks ON on input line 4688.
\c@Hy@tempcnt=\count111
(/usr/share/texlive/texmf-dist/tex/latex/url/url.sty
\Urlmuskip=\muskip11
Package: url 2013/09/16  ver 3.4  Verb mode for urls, etc.
)
LaTeX Info: Redefining \url on input line 5041.
\XeTeXLinkMargin=\dimen126
\Fld@menulength=\count112
\Field@Width=\dimen127
\Fld@charsize=\dimen128
Package hyperref Info: Hyper figures OFF on input line 6295.
Package hyperref Info: Link nesting OFF on input line 6300.
Package hyperref Info: Hyper index ON on input line 6303.
Package hyperref Info: backreferencing OFF on input line 6310.
Package hyperref Info: Link coloring ON on input line 6313.
Package hyperref Info: Link coloring with OCG OFF on input line 6320.
Package hyperref Info: PDF/A mode OFF on input line 6325.
LaTeX Info: Redefining \ref on input line 6365.
LaTeX Info: Redefining \pageref on input line 6369.
\Hy@abspage=\count113
\c@Item=\count114
)

Package hyperref Message: Driver: hpdftex.

(/usr/share/texlive/texmf-dist/tex/latex/hyperref/hpdftex.def
File: hpdftex.def 2012/11/06 v6.83m Hyperref driver for pdfTeX
\Fld@listcount=\count115
\c@bookmark@seq@number=\count116
(/usr/share/texlive/texmf-dist/tex/latex/oberdiek/rerunfilecheck.sty
Package: rerunfilecheck 2011/04/15 v1.7 Rerun checks for auxiliary files (HO)
Package uniquecounter Info: New unique counter `rerunfilecheck' on input line 2
82.
)
\Hy@SectionHShift=\skip58
) (./7825-t.aux
LaTeX Font Info:    Try loading font information for LGR+cmr on input line 23.
(/usr/share/texlive/texmf-dist/tex/latex/cbfonts-fd/lgrcmr.fd
File: lgrcmr.fd 2013/09/01 v1.0 Greek European Computer Regular
))
\openout1 = `7825-t.aux'.

LaTeX Font Info:    Checking defaults for OML/cmm/m/it on input line 371.
LaTeX Font Info:    ... okay on input line 371.
LaTeX Font Info:    Checking defaults for T1/cmr/m/n on input line 371.
LaTeX Font Info:    ... okay on input line 371.
LaTeX Font Info:    Checking defaults for OT1/cmr/m/n on input line 371.
LaTeX Font Info:    ... okay on input line 371.
LaTeX Font Info:    Checking defaults for OMS/cmsy/m/n on input line 371.
LaTeX Font Info:    ... okay on input line 371.
LaTeX Font Info:    Checking defaults for OMX/cmex/m/n on input line 371.
LaTeX Font Info:    ... okay on input line 371.
LaTeX Font Info:    Checking defaults for U/cmr/m/n on input line 371.
LaTeX Font Info:    ... okay on input line 371.
LaTeX Font Info:    Checking defaults for LGR/cmr/m/n on input line 371.
LaTeX Font Info:    ... okay on input line 371.
LaTeX Font Info:    Checking defaults for PD1/pdf/m/n on input line 371.
LaTeX Font Info:    ... okay on input line 371.
(/usr/share/texlive/texmf-dist/tex/context/base/supp-pdf.mkii
[Loading MPS to PDF converter (version 2006.09.02).]
\scratchcounter=\count117
\scratchdimen=\dimen129
\scratchbox=\box28
\nofMPsegments=\count118
\nofMParguments=\count119
\everyMPshowfont=\toks23
\MPscratchCnt=\count120
\MPscratchDim=\dimen130
\MPnumerator=\count121
\makeMPintoPDFobject=\count122
\everyMPtoPDFconversion=\toks24
)
*geometry* driver: auto-detecting
*geometry* detected driver: pdftex
*geometry* verbose mode - [ preamble ] result:
* driver: pdftex
* paper: <default>
* layout: <same size as paper>
* layoutoffset:(h,v)=(0.0pt,0.0pt)
* hratio: 1:1
* modes: includehead includefoot twoside 
* h-part:(L,W,R)=(9.03375pt, 307.14749pt, 9.03375pt)
* v-part:(T,H,B)=(1.26749pt, 466.58623pt, 1.90128pt)
* \paperwidth=325.215pt
* \paperheight=469.75499pt
* \textwidth=307.14749pt
* \textheight=404.71243pt
* \oddsidemargin=-63.23624pt
* \evensidemargin=-63.23624pt
* \topmargin=-71.0025pt
* \headheight=12.0pt
* \headsep=19.8738pt
* \topskip=12.0pt
* \footskip=30.0pt
* \marginparwidth=98.0pt
* \marginparsep=7.0pt
* \columnsep=10.0pt
* \skip\footins=10.8pt plus 4.0pt minus 2.0pt
* \hoffset=0.0pt
* \voffset=0.0pt
* \mag=1000
* \@twocolumnfalse
* \@twosidetrue
* \@mparswitchtrue
* \@reversemarginfalse
* (1in=72.27pt=25.4mm, 1cm=28.453pt)

\AtBeginShipoutBox=\box29
(/usr/share/texlive/texmf-dist/tex/latex/graphics/color.sty
Package: color 2005/11/14 v1.0j Standard LaTeX Color (DPC)
(/usr/share/texlive/texmf-dist/tex/latex/latexconfig/color.cfg
File: color.cfg 2007/01/18 v1.5 color configuration of teTeX/TeXLive
)
Package color Info: Driver file: pdftex.def on input line 130.
)
Package hyperref Info: Link coloring ON on input line 371.
(/usr/share/texlive/texmf-dist/tex/latex/hyperref/nameref.sty
Package: nameref 2012/10/27 v2.43 Cross-referencing by name of section
(/usr/share/texlive/texmf-dist/tex/generic/oberdiek/gettitlestring.sty
Package: gettitlestring 2010/12/03 v1.4 Cleanup title references (HO)
)
\c@section@level=\count123
)
LaTeX Info: Redefining \ref on input line 371.
LaTeX Info: Redefining \pageref on input line 371.
LaTeX Info: Redefining \nameref on input line 371.
(./7825-t.out) (./7825-t.out)
\@outlinefile=\write3
\openout3 = `7825-t.out'.


Overfull \hbox (28.60733pt too wide) in paragraph at lines 378--378
[]\OT1/cmtt/m/n/8 The Project Gutenberg EBook of Geometrical Solutions Derived 
from Mechanics, by[] 
 []


Overfull \hbox (7.35703pt too wide) in paragraph at lines 381--381
[]\OT1/cmtt/m/n/8 This eBook is for the use of anyone anywhere in the United St
ates and most[] 
 []


Overfull \hbox (7.35703pt too wide) in paragraph at lines 383--383
[]\OT1/cmtt/m/n/8 whatsoever.  You may copy it, give it away or re-use it under
 the terms of[] 
 []


Overfull \hbox (15.85715pt too wide) in paragraph at lines 385--385
[]\OT1/cmtt/m/n/8 www.gutenberg.org.  If you are not located in the United Stat
es, you'll have[] 
 []


Overfull \hbox (28.60733pt too wide) in paragraph at lines 386--386
[]\OT1/cmtt/m/n/8 to check the laws of the country where you are located before
 using this ebook.[] 
 []

LaTeX Font Info:    Try loading font information for U+msa on input line 403.
(/usr/share/texlive/texmf-dist/tex/latex/amsfonts/umsa.fd
File: umsa.fd 2013/01/14 v3.01 AMS symbols A
)
LaTeX Font Info:    Try loading font information for U+msb on input line 403.
(/usr/share/texlive/texmf-dist/tex/latex/amsfonts/umsb.fd
File: umsb.fd 2013/01/14 v3.01 AMS symbols B
) [1

{/var/lib/texmf/fonts/map/pdftex/updmap/pdftex.map}] [2] [1


] [2] [1



] [2] [3] [4] [5] [6] [7] [8] [9

] [10] [11] [12] <./images/fig01.png, id=187, 397.485pt x 349.305pt>
File: ./images/fig01.png Graphic file (type png)
<use ./images/fig01.png>
Package pdftex.def Info: ./images/fig01.png used on input line 867.
(pdftex.def)             Requested size: 202.71846pt x 178.14851pt.
[13] [14 <./images/fig01.png>] [15] <./images/fig02.png, id=204, 375.4025pt x 3
49.305pt>
File: ./images/fig02.png Graphic file (type png)
<use ./images/fig02.png>
Package pdftex.def Info: ./images/fig02.png used on input line 908.
(pdftex.def)             Requested size: 202.71846pt x 188.62721pt.
[16 <./images/fig02.png>] [17] [18] [19] <./images/fig03.png, id=228, 334.24875
pt x 362.35374pt>
File: ./images/fig03.png Graphic file (type png)
<use ./images/fig03.png>
Package pdftex.def Info: ./images/fig03.png used on input line 1045.
(pdftex.def)             Requested size: 202.71846pt x 219.7693pt.
[20] [21 <./images/fig03.png>] [22] <./images/fig04.png, id=244, 298.11375pt x 
277.035pt>
File: ./images/fig04.png Graphic file (type png)
<use ./images/fig04.png>
Package pdftex.def Info: ./images/fig04.png used on input line 1118.
(pdftex.def)             Requested size: 202.71846pt x 188.38553pt.
[23] [24] [25 <./images/fig04.png>] <./images/fig05.png, id=261, 313.17pt x 336
.25626pt>
File: ./images/fig05.png Graphic file (type png)
<use ./images/fig05.png>
Package pdftex.def Info: ./images/fig05.png used on input line 1175.
(pdftex.def)             Requested size: 202.71846pt x 217.6712pt.
[26] [27 <./images/fig05.png>] [28] <./images/fig06.png, id=279, 201.75375pt x 
373.395pt>
File: ./images/fig06.png Graphic file (type png)
<use ./images/fig06.png>
Package pdftex.def Info: ./images/fig06.png used on input line 1246.
(pdftex.def)             Requested size: 153.57375pt x 284.22755pt.
[29] [30 <./images/fig06.png>] <./images/fig07.png, id=292, 404.51125pt x 416.5
5624pt>
File: ./images/fig07.png Graphic file (type png)
<use ./images/fig07.png>
Package pdftex.def Info: ./images/fig07.png used on input line 1285.
(pdftex.def)             Requested size: 202.71846pt x 208.75432pt.
[31] [32 <./images/fig07.png>] <./images/fig08.png, id=306, 279.0425pt x 450.68
375pt>
File: ./images/fig08.png Graphic file (type png)
<use ./images/fig08.png>
Package pdftex.def Info: ./images/fig08.png used on input line 1332.
(pdftex.def)             Requested size: 153.57375pt x 248.04185pt.
[33] [34 <./images/fig08.png>] [35] [36] [37] <./images/fig09.png, id=339, 301.
125pt x 303.1325pt>
File: ./images/fig09.png Graphic file (type png)
<use ./images/fig09.png>
Package pdftex.def Info: ./images/fig09.png used on input line 1460.
(pdftex.def)             Requested size: 138.21542pt x 139.14198pt.
<./images/fig10.png, id=340, 246.9225pt x 239.89626pt>
File: ./images/fig10.png Graphic file (type png)
<use ./images/fig10.png>
Package pdftex.def Info: ./images/fig10.png used on input line 1460.
(pdftex.def)             Requested size: 138.21542pt x 134.28241pt.
[38] [39 <./images/fig09.png> <./images/fig10.png>] [40] <./images/fig11.png, i
d=361, 242.9075pt x 234.8775pt>
File: ./images/fig11.png Graphic file (type png)
<use ./images/fig11.png>
Package pdftex.def Info: ./images/fig11.png used on input line 1527.
(pdftex.def)             Requested size: 202.71846pt x 196.01628pt.
[41] <./images/fig12.png, id=370, 240.9pt x 237.88875pt>
File: ./images/fig12.png Graphic file (type png)
<use ./images/fig12.png>
Package pdftex.def Info: ./images/fig12.png used on input line 1568.
(pdftex.def)             Requested size: 202.71846pt x 200.18816pt.
[42 <./images/fig11.png>] [43] [44 <./images/fig12.png>] [45] [46] [1



]
Overfull \hbox (3.10696pt too wide) in paragraph at lines 1706--1706
[]\OT1/cmtt/m/n/8 1.C.  The Project Gutenberg Literary Archive Foundation ("the
 Foundation"[] 
 []


Overfull \hbox (3.10696pt too wide) in paragraph at lines 1711--1711
[]\OT1/cmtt/m/n/8 located in the United States, we do not claim a right to prev
ent you from[] 
 []


Overfull \hbox (3.10696pt too wide) in paragraph at lines 1716--1716
[]\OT1/cmtt/m/n/8 freely sharing Project Gutenberg-tm works in compliance with 
the terms of[] 
 []

[2] [3]
Overfull \hbox (3.10696pt too wide) in paragraph at lines 1779--1779
[]\OT1/cmtt/m/n/8 posted on the official Project Gutenberg-tm web site (www.gut
enberg.org),[] 
 []

[4] [5] [6] [7] [8]
Package atveryend Info: Empty hook `BeforeClearDocument' on input line 2008.
[9]
Package atveryend Info: Empty hook `AfterLastShipout' on input line 2008.
(./7825-t.aux)
Package atveryend Info: Executing hook `AtVeryEndDocument' on input line 2008.

 *File List*
    book.cls    2007/10/19 v1.4h Standard LaTeX document class
   leqno.clo    1998/08/17 v1.1c Standard LaTeX option (left equation numbers)
    bk12.clo    2007/10/19 v1.4h Standard LaTeX file (size option)
inputenc.sty    2008/03/30 v1.1d Input encoding file
  latin1.def    2008/03/30 v1.1d Input encoding file
  ifthen.sty    2001/05/26 v1.1c Standard LaTeX ifthen package (DPC)
   babel.sty    2013/12/03 3.9h The Babel package
   greek.ldf    2013/12/03 v1.8a Greek support for the babel system
  lgrenc.def    2013/07/16 v0.9 LGR Greek font encoding definitions
greek-fontenc.def    2013/11/28 v0.11 Common Greek font encoding definitions
 english.ldf    2012/08/20 v3.3p English support from the babel system
 amsmath.sty    2013/01/14 v2.14 AMS math features
 amstext.sty    2000/06/29 v2.01
  amsgen.sty    1999/11/30 v2.0
  amsbsy.sty    1999/11/29 v1.2d
  amsopn.sty    1999/12/14 v2.01 operator names
 amssymb.sty    2013/01/14 v3.01 AMS font symbols
amsfonts.sty    2013/01/14 v3.01 Basic AMSFonts support
   alltt.sty    1997/06/16 v2.0g defines alltt environment
indentfirst.sty    1995/11/23 v1.03 Indent first paragraph (DPC)
graphicx.sty    1999/02/16 v1.0f Enhanced LaTeX Graphics (DPC,SPQR)
  keyval.sty    1999/03/16 v1.13 key=value parser (DPC)
graphics.sty    2009/02/05 v1.0o Standard LaTeX Graphics (DPC,SPQR)
    trig.sty    1999/03/16 v1.09 sin cos tan (DPC)
graphics.cfg    2010/04/23 v1.9 graphics configuration of TeX Live
  pdftex.def    2011/05/27 v0.06d Graphics/color for pdfTeX
infwarerr.sty    2010/04/08 v1.3 Providing info/warning/error messages (HO)
 ltxcmds.sty    2011/11/09 v1.22 LaTeX kernel commands for general use (HO)
    calc.sty    2007/08/22 v4.3 Infix arithmetic (KKT,FJ)
fancyhdr.sty    
geometry.sty    2010/09/12 v5.6 Page Geometry
   ifpdf.sty    2011/01/30 v2.3 Provides the ifpdf switch (HO)
  ifvtex.sty    2010/03/01 v1.5 Detect VTeX and its facilities (HO)
 ifxetex.sty    2010/09/12 v0.6 Provides ifxetex conditional
hyperref.sty    2012/11/06 v6.83m Hypertext links for LaTeX
hobsub-hyperref.sty    2012/05/28 v1.13 Bundle oberdiek, subset hyperref (HO)
hobsub-generic.sty    2012/05/28 v1.13 Bundle oberdiek, subset generic (HO)
  hobsub.sty    2012/05/28 v1.13 Construct package bundles (HO)
ifluatex.sty    2010/03/01 v1.3 Provides the ifluatex switch (HO)
 intcalc.sty    2007/09/27 v1.1 Expandable calculations with integers (HO)
etexcmds.sty    2011/02/16 v1.5 Avoid name clashes with e-TeX commands (HO)
kvsetkeys.sty    2012/04/25 v1.16 Key value parser (HO)
kvdefinekeys.sty    2011/04/07 v1.3 Define keys (HO)
pdftexcmds.sty    2011/11/29 v0.20 Utility functions of pdfTeX for LuaTeX (HO)
pdfescape.sty    2011/11/25 v1.13 Implements pdfTeX's escape features (HO)
bigintcalc.sty    2012/04/08 v1.3 Expandable calculations on big integers (HO)
  bitset.sty    2011/01/30 v1.1 Handle bit-vector datatype (HO)
uniquecounter.sty    2011/01/30 v1.2 Provide unlimited unique counter (HO)
letltxmacro.sty    2010/09/02 v1.4 Let assignment for LaTeX macros (HO)
 hopatch.sty    2012/05/28 v1.2 Wrapper for package hooks (HO)
xcolor-patch.sty    2011/01/30 xcolor patch
atveryend.sty    2011/06/30 v1.8 Hooks at the very end of document (HO)
atbegshi.sty    2011/10/05 v1.16 At begin shipout hook (HO)
refcount.sty    2011/10/16 v3.4 Data extraction from label references (HO)
 hycolor.sty    2011/01/30 v1.7 Color options for hyperref/bookmark (HO)
 auxhook.sty    2011/03/04 v1.3 Hooks for auxiliary files (HO)
kvoptions.sty    2011/06/30 v3.11 Key value format for package options (HO)
  pd1enc.def    2012/11/06 v6.83m Hyperref: PDFDocEncoding definition (HO)
hyperref.cfg    2002/06/06 v1.2 hyperref configuration of TeXLive
     url.sty    2013/09/16  ver 3.4  Verb mode for urls, etc.
 hpdftex.def    2012/11/06 v6.83m Hyperref driver for pdfTeX
rerunfilecheck.sty    2011/04/15 v1.7 Rerun checks for auxiliary files (HO)
  lgrcmr.fd    2013/09/01 v1.0 Greek European Computer Regular
supp-pdf.mkii
   color.sty    2005/11/14 v1.0j Standard LaTeX Color (DPC)
   color.cfg    2007/01/18 v1.5 color configuration of teTeX/TeXLive
 nameref.sty    2012/10/27 v2.43 Cross-referencing by name of section
gettitlestring.sty    2010/12/03 v1.4 Cleanup title references (HO)
  7825-t.out
  7825-t.out
    umsa.fd    2013/01/14 v3.01 AMS symbols A
    umsb.fd    2013/01/14 v3.01 AMS symbols B
./images/fig01.png
./images/fig02.png
./images/fig03.png
./images/fig04.png
./images/fig05.png
./images/fig06.png
./images/fig07.png
./images/fig08.png
./images/fig09.png
./images/fig10.png
./images/fig11.png
./images/fig12.png
 ***********

Package atveryend Info: Executing hook `AtEndAfterFileList' on input line 2008.

Package rerunfilecheck Info: File `7825-t.out' has not changed.
(rerunfilecheck)             Checksum: 4A75607023C033ACCE3941ED3C3BF718;2021.
Package atveryend Info: Empty hook `AtVeryVeryEnd' on input line 2008.
 ) 
Here is how much of TeX's memory you used:
 8050 strings out of 493304
 117723 string characters out of 6139872
 218823 words of memory out of 5000000
 11246 multiletter control sequences out of 15000+600000
 17147 words of font info for 63 fonts, out of 8000000 for 9000
 957 hyphenation exceptions out of 8191
 28i,12n,43p,485b,525s stack positions out of 5000i,500n,10000p,200000b,80000s
</usr/share/texlive/texmf-dist/fonts/type1/public/amsfonts/cm/cmcsc10.pfb></u
sr/share/texlive/texmf-dist/fonts/type1/public/amsfonts/cm/cmmi12.pfb></usr/sha
re/texlive/texmf-dist/fonts/type1/public/amsfonts/cm/cmr10.pfb></usr/share/texl
ive/texmf-dist/fonts/type1/public/amsfonts/cm/cmr12.pfb></usr/share/texlive/tex
mf-dist/fonts/type1/public/amsfonts/cm/cmr17.pfb></usr/share/texlive/texmf-dist
/fonts/type1/public/amsfonts/cm/cmr7.pfb></usr/share/texlive/texmf-dist/fonts/t
ype1/public/amsfonts/cm/cmr8.pfb></usr/share/texlive/texmf-dist/fonts/type1/pub
lic/amsfonts/cm/cmsy10.pfb></usr/share/texlive/texmf-dist/fonts/type1/public/am
sfonts/cm/cmti10.pfb></usr/share/texlive/texmf-dist/fonts/type1/public/amsfonts
/cm/cmti12.pfb></usr/share/texlive/texmf-dist/fonts/type1/public/amsfonts/cm/cm
tt10.pfb></usr/share/texlive/texmf-dist/fonts/type1/public/amsfonts/cm/cmtt8.pf
b></usr/share/texlive/texmf-dist/fonts/type1/public/cbfonts/grmn1000.pfb></usr/
share/texlive/texmf-dist/fonts/type1/public/cbfonts/grmn1200.pfb>
Output written on 7825-t.pdf (59 pages, 427615 bytes).
PDF statistics:
 515 PDF objects out of 1000 (max. 8388607)
 423 compressed objects within 5 object streams
 136 named destinations out of 1000 (max. 500000)
 245 words of extra memory for PDF output out of 10000 (max. 10000000)

